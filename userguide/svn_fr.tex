\section{Support de SVN}\label{sec:svn}

\hint{NdT : Cette extension est traduite ici, mais est obsolète. Vous avez donc de grandes chances de ne plus la trouver dans les versions récentes de \codeblocks.}

Le support du système de contrôle de version SVN est inclus dans l'extension \codeblocks TortoiseSVN. Via le menu \menu{TortoiseSVN,Plugin settings} vous pouvez configurer les commandes svn accessibles dans l'onglet \menu{Integration}.

\begin{description}
\item[Menu intégration] Ajoute une entrée TortoiseSVN dans la barre de menu avec différents paramétrages.
\item[Project manager] Active les commandes TortoiseSVN du menu de contexte de la gestion de projet.
\item[Editor] Active les commandes TortoiseSVN du menu de contexte de l'éditeur.
\end{description}

Dans la configuration de l'extension vous pouvez choisir quelles sont les commandes svn qui sont accessibles dans le menu principal ou le menu de contexte. L'onglet intégration fournit une entrée \menu{Edit main menu} et \menu{Edit popup menu} pour paramétrer ces commandes.

\hint{L'Explorateur de fichiers dans \codeblocks utilise différentes icônes superposées afin d'indiquer l'état de svn. Les commandes de TortoiseSVN sont incluses dans le menu de contexte.}
