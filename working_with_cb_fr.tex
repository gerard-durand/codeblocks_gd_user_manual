\chapter{Travailler avec \codeblocks}

Ce chapitre traite de quelques connaissances de base pour pouvoir travailler avec \codeblocks. Quelques paragraphes, ici directement repris du Wiki font double emploi mais avec une présentation un peu différente de ce qu'il y a dans le premier chapitre.

\begin{BUILDPROCESS}
\section{Le processus de génération de \codeblocks}\label{sec:build_process}

Dans ces pages, le processus de génération est expliqué en détail. On y voit ce qui se passe en arrière-plan et "quand". Je vous souhaite une intéressante lecture :).
 
\subsection{Étapes successives de la Génération}

Comme vous l'avez peut-être déjà compris, \codeblocks ne lance pas au hasard les commandes de génération, mais effectue plutôt une séquence d'étapes bien pensées et préparées. Mais avant tout, regardons les différents composants qui sont utilisés lors d'une génération:

\begin{description}
\item [Espace de Travail :] contient un ou plusieurs projets (dénommé aussi workspace, comme en anglais)
\item [Projet :] contient une ou plusieurs cibles de génération. Il contient également les fichiers de projet.
\item [Cible de génération :] ce sont les variantes de projet qui lui-sont assignés, et qui seront générées par groupes afin de produire une sortie binaire. Cette sortie peut être soit un exécutable, une librairie  dynamique ou statique. \textbf{Note :} Il existe un type de cible de génération qui ne produit pas directement une sortie binaire mais se contente plutôt de seulement réaliser des étapes de pre/post génération (qui peuvent générer de façon externe une sortie binaire).
\end{description}

Décomposons ces sujets en sections et expliquons-les en détail.

\subsection{Espace de Travail}

Un espace de travail (ou Workspace) est un conteneur (celui de plus haut niveau) utilisé pour organiser vos projets. Comme il ne peut y avoir qu'un seul espace de travail ouvert à la fois, il n'y a pas d'ordre spécifique les concernant. Un seul espace, donc il suffit de le générer ;).

Utilisez le menu \menu{Générer,Générer l'espace de travail} pour générer l'espace de travail (c.à.d. tous les projets qui y sont contenus). 

\subsection{Projets}

C'est ici que les choses deviennent intéressantes :).

L'ordre de génération des projets est différent suivant que l'utilisateur a indiqué s'il y a des dépendances ou pas entre les projets. Alors, allons-y...

\genterm{Sans dépendances inter-projets}

Dans ce cas, les projets sont générés dans l'ordre d'apparition, du haut vers le bas. Dans la plupart des projets cependant (sauf ceux du genre "hello world"), vous allez vouloir créer des dépendances entre projets.

\genterm{Utilisation de dépendances entre projets}

Les dépendances de projets sont une façon simple d'indiquer à \codeblocks qu'un projet donné "dépend" d'un autre (toujours dans le même espace de travail).

Alors imaginons que, dans votre espace de travail, vous avez un projet de librairie et un projet d'exécutable qui dépend de cette librairie. Vous pouvez alors (et devez) informer \codeblocks de cette dépendance. Pour faire cela, vous sélectionnez \menu{Projet,Propriétés} et cliquez sur le bouton des  "Dépendances de Projet...".

\textit{Veuillez noter que les informations de dépendances sont enregistrées dans le fichier de l'espace de travail, et non dans un fichier projet car les dépendances se font entre deux projets à l'intérieur d'un même espace de travail.}

\figures[H][width=.55\columnwidth]{Project_deps}{Configuration de dépendances de projet}

C'est très facile d'utiliser ce dialogue. Sélectionnez le projet sur lequel vous voulez ajouter une dépendance et cochez la case sur tous les projets dont ce projet de base dépend. Cela aura pour conséquence que tous les projets que vous avez coché seront générés avant le projet qui en dépend, assurant ainsi une génération synchronisée.

\textbf{Astuce :} Vous n'avez pas à fermer ce dialogue et lancer d'autres propriétés de projets de nouveau pour configurer leurs dépendances. Vous pouvez configurer toutes les dépendances de projets depuis cette même boîte de dialogue. Sélectionnez simplement un autre projet dans le menu déroulant :).

Quelques choses à noter :

\begin{itemize}
\item Les dépendances sont configurées directement ou indirectement. Si le projet A dépend directement du projet B et que le projet B dépend du projet C, alors le projet A dépend également du projet C, mais indirectement.
\item \codeblocks est suffisamment intelligent pour vérifier s'il y a des dépendances circulaires et donc les interdire. Une dépendance circulaire est obtenue quand un projet A dépend directement ou indirectement d'un projet B et que le projet B dépend directement ou indirectement du projet A.
\item Les dépendances prennent effet soit lors de la génération d'un espace de travail entier soit sur un projet seul. Dans ce cas, seules les dépendances nécessaires au projet que vous êtes en train de générer seront aussi générées.
\end{itemize}

\subsection{Génération de Cibles}

L'ordre de génération des cibles dépend de deux choses.

\begin{enumerate}
\item Si l'utilisateur a sélectionné une cible particulière dans le menu déroulant de la barre de compilation, alors seule cette cible sera générée. Si des dépendances de projet ont été configurées pour le projet contenant cette cible, tous les projets dépendants génèreront aussi leur cible sous le même nom. Si une telle cible n'existe pas, on passe au projet suivant.
\item Si la cible virtuelle "All" est sélectionnée, alors toutes les cibles dans le projet (et tous les projets dépendants) sont générés dans l'ordre du haut vers le bas. Il y a deux exceptions à cela :
    \begin{itemize}
    \item Une cible n'est pas générée par "All" si l'option de cible (dans la page des propriétés du projet "Cibles de génération") "Générer cette cible par All" n'est pas sélectionnée.
    \item Si aucune cible du projet n'a de sélectionnées l'option ci-dessus, alors la cible "All" n'apparait pas dans la liste.
    \end{itemize}
\end{enumerate}

\subsection{Phase de Preprocessing}

Avant que le processus de génération démarre (c.à.d. commence l'exécution des commandes de compilation/édition de liens), une étape de preprocessing est lancée pour générer toutes les lignes de commandes du processus complet de génération. Cette étape place dans un cache la plupart des informations qu'elle génère, ce qui a pour effet de rendre les générations suivantes plus rapides.

Cette étape lance aussi tout script de génération qui y est attaché.


\subsection{Commandes réelles d'exécution}

C'est l'étape, du point de vue de l'utilisateur, où le processus de génération commence réellement. Les fichiers commencent à être compilés et au final liés entre eux pour générer les diverses sorties binaires définies par les cibles de génération.

Dans cette étape sont aussi exécutées les cibles de pré-génération et de post-génération.


\subsection{Étape de pré-génération et post-génération}

Ce sont des commandes qui peuvent être configurées au niveau projet et/ou au niveau cible. Ce sont des commandes Shell qui par exemple copient des fichiers ou toute autre opération que vous pouvez réaliser par les commandes Shell habituelles de l'OS.

Les variables spécifiées dans le paragraphe Expansion de Variables (\pxref{sec:variables_types}) peuvent être utilisées dans les scripts afin de récupérer des informations comme le répertoire de sortie, le répertoire de projet, le type de cible ou autres.

Vous avez ci-dessous le déroulé dans l'ordre d'exécution des étapes de pré/post génération d'un projet imaginaire avec 2 cibles (Debug/Release) :

\begin{enumerate}
\item Étapes de pré-génération du Projet
    \begin{enumerate}
    \item Target "Debug" étapes de pré-génération
    \item Target "Debug" compilation des fichiers
    \item Target "Debug" édition de liens des fichiers et génération de la sortie binaire
    \item Target "Debug" étapes de post-génération (voir les notes ci-dessous)
    \item Target "Release" étapes de pré-génération
    \item Target "Release" compilation des fichiers
    \item Target "Release" édition de liens des fichiers et génération de la sortie binaire
    \item Target "Release" étapes de post-génération (voir les notes ci-dessous)
    \end{enumerate}
\item Étapes de post-génération du Projet
\end{enumerate}

J'espère que c'est suffisamment clair :)

\hint{Les étapes de Pré-génération sont toujours exécutées. Les étapes de Post-génération ne seront exécutées que si le projet/cible auxquelles elles sont rattachées n'est pas à jour (c.à.d. en train d'être généré). Vous pouvez changer cela en sélectionnant "Toujours exécuter, même si la cible est à jour" dans les options de génération.}

\genterm{Exemples de Script}

Script de Post-génération qui copie le fichier de sortie dans un répertoire \file{C:\osp Program\osp bin} sous Windows : 

\begin{lstlisting}
cmd /c copy "$(PROJECT_DIR)$(TARGET_OUTPUT_FILE)" "C:\Program\bin"
\end{lstlisting}

Exécution du script bash "copyresources.sh" sous Linux :

\begin{lstlisting}
/bin/sh copyresources.sh
\end{lstlisting}

Création d'un nouveau répertoire dans le répertoire de sortie :

\begin{lstlisting}
mkdir $(TARGET_OUTPUT_DIR)/data
\end{lstlisting}  

\end{BUILDPROCESS}

\begin{CREATEPROJECT}
\section{Création d'un Nouveau Projet}\label{sec:create_project}

Ces pages sont un guide sur les fonctionnalités de base (et quelques intermédiaires) pour la création et la modification d'un projet \codeblocks. Si c'est votre première exéprience avec \codeblocks, alors ceci est un bon point de départ. 
 
\subsection{L'assistant de Projet}

Lancez l'assistant de Projet par \menu{Fichier,Nouveau,Projet...} afin de démarrer un nouveau projet. Ici, il y a plusieurs modèles pré-configurés pour divers types de projets, incluant l'option de création de modèles personnalisés. Sélectionnez  \textbf{Console application}, car c'est le plus commun pour un usage général, puis cliquez sur \textbf{Aller} ou \textbf{Go}. 

\screenshot{ProjectWizard}{L'assistant de Projet}

\hint{Un texte en rouge au lieu d'un texte en noir sous n'importe quelle icône signifie que l'on utilise un script assistant personnalisé.}

L'assistant Application console apparaitra ensuite. Continuez dans les menus, en sélectionnant \textbf{C++} quand on vous demandera le langage. Sur l'écran suivant, donnez un nom au projet et tapez ou sélectionnez un répertoire de destination. Comme on voit ci-dessous, \codeblocks génèrera les entrées restantes à partir de ces deux là. 

\figures[H][width=.6\columnwidth]{ConsoleApplication}{Application Console}

Finalement, l'assistant vous demandera si ce projet doit utiliser le compilateur par défaut (normalement GCC) et les deux générations par défaut : \textbf{Debug} et \textbf{Release}. Toutes ces configurations sont correctes. Appuyez sur Terminer et le projet va se générer. La fenêtre principale va se griser, mais ce n'est pas un problème, le fichier source doit encore être ouvert. Dans l'onglet \textbf{Projets} du panneau de \textbf{Gestion} sur la gauche, dépliez les répertoires et double-cliquez sur le fichier source \textbf{main.cpp} afin de l'ouvrir dans l'éditeur. 

\figures[H][width=.45\columnwidth]{SelectSource}{Selection d'un fichier Source}
Ce fichier contient le code standard suivant.

main.cpp 
\begin{lstlisting}
#include <iostream>
using namespace std;
int main()
{
    cout << "Hello world!" << endl;
    return 0;
}
\end{lstlisting}

\subsection{Changer la composition du fichier}

Un simple fichier source est d'un intérêt limité dans des programmes plus ou moins complexes. Pour traiter cela, \codeblocks possède plusieurs méthodes très simples permettant d'ajouter des fichiers supplémentaires au projet.

\genterm{Ajout d'un fichier vide}

Dans cet exemple, nous allons isoler la fonction

main.cpp
\begin{lstlisting}
    cout << "Hello world!" << endl;
\end{lstlisting}

dans un fichier séparé.

\hint{Généralement, c'est un mauvais style de programmation que de créer une fonction dans un aussi petit fichier ; ici, ce n'est fait qu'à titre d'exemple.}

Pour ajouter un nouveau fichier au projet, appelez l'assistant de modèle de fichier soit par le \menu{Fichier,Nouveau,Fichier...} soit par \menu{Barre d'outils principale,Nouveau fichier (bouton),Fichier...} 
Utilisez le menu \menu{Générer,Générer l'espace de travail} pour générer un espace de travail (c.a.d tous les projets qui y sont contenus). 

\figures[H][width=1.1\columnwidth]{NewFile}{Nouveau Fichier}
Sélectionnez la source en \textbf{C/C++} et cliquez sur \textbf{Aller} (ou \textbf{Go}). Continuez dans les dialogues suivants tout comme lors de la création du projet original, en sélectionnant \textbf{C++} quand on vous demandera le langage. La page finale vous présentera plusieurs options. La première boîte détermine le nouveau nom du fichier et l'emplacement (comme noté, le chemin complet est requis). Vous pouvez utiliser en option le bouton \codeline{...} pour afficher une fenêtre de navigateur de fichiers pour enregistrer l'emplacement du fichier. En cochant \textbf{Ajouter le fichier au projet actif} vous enregistrerez le nom du fichier dans le répertoire \textbf{Sources} de l'onglet \textbf{Projets} du panneau de \textbf{Gestion}. En cochant une ou plusieurs cibles de génération vous informerez \codeblocks que le fichier devra être compilé puis de faire l'édition de liens pour la(les) cible(s) sélectionnée(s). Ce peut être utile si, par exemple, le fichier contient du code spécifique de débogage, car cela permettra l'inclusion dans (ou l'exclusion de) la (les) cible(s) de génération appropriée(s). Dans cet exemple, toutefois, la fonction hello est indispensable, et donc requise dans toutes les sources. Par conséquent, sélectionnez toutes les cases et cliquez sur \textbf{Terminer} pour générer le fichier.

\figures[H][width=.55\columnwidth]{Hello}{Configurations du Programme Hello }
Le fichier nouvellement créé devrait s'ouvrir automatiquement ; si ce n'est pas le cas, ouvrez le en double-cliquant sur ce fichier dans l'onglet \textbf{Projets} du panneau de \textbf{Gestion}. Ajoutez-y maintenant le code de la fonction que \textbf{main.cpp} appelera.

hello.cpp
\begin{lstlisting}
#include <iostream>
using namespace std;
  
void hello()
{
    cout << "Hello world!" << endl;
} 
\end{lstlisting}

\genterm{Ajout d'un fichier déjà existant}

Maintenant que la fonction \textbf{hello()} est dans un fichier séparé, la fonction doit être déclarée dans \textbf{main.cpp} pour pouvoir être utilisée. Lancez un éditeur de texte (par exemple Notepad ou Gedit), et ajoutez le code suivant :

hello.h 
\begin{lstlisting}
#ifndef HELLO_H_INCLUDED
#define HELLO_H_INCLUDED
     
void hello();
     
#endif // HELLO_H_INCLUDED
\end{lstlisting}

Enregistrez ce fichier en tant que fichier d'en-tête (\textbf{hello.h}) dans le même répertoire que les autres fichiers source de ce projet. Revenez dans \codeblocks, cliquez sur \menu{Projet,Ajouter des  fichiers...} pour ouvrir un navigateur de fichiers. Là, vous pouvez sélectionner un ou plusieurs fichiers (en utilisant les combinations de \textit{Ctrl} et \textit{Maj}). (L'option \menu{Projet,Ajouter des fichiers récursivement...} va chercher dans tous les sous-répertoires d'un répertoire donné, en sélectionnant les fichiers adéquats à inclure.) Sélectionnez \textbf{hello.h}, et cliquez sur \textbf{Open} pour obtenir le dialoque vous demandant dans quelles cibles le(les) fichiers doivent appartenir. Dans cet exemple, sélectionnez les deux cibles. 

\figures[H][width=.6\columnwidth]{TargetBelonging}{Appartenance aux cibles de génération}

\hint{Si le projet en cours n'a qu'une seule cible, on ne passera pas par ce dialogue.}

De retour dans le source principal (\textbf{main.cpp}), incluez le fichier d'en-tête et replacez la fonction \codeline{cout} afin de se conformer à la nouvelle configuration du projet.

main.cpp
\begin{lstlisting}
#include "hello.h"

int main()
{
    hello();
    return 0;
}
\end{lstlisting}

Pressez Ctrl-F9, \menu{Générer,Générer}, ou \menu{Barre d'outils du Compilateur,Générer (bouton - la roue dentée)} pour compiler le projet. Si la sortie suivante s'affiche dans le journal de génération (dans le panneau du bas) c'est que toutes les étapes se sont correctement déroulées.

\begin{lstlisting}
-------------- Build: Debug in HelloWorld ---------------

Compiling: main.cpp
Compiling: hello.cpp
Linking console executable: bin\Debug\HelloWorld.exe
Output size is 923.25 KB
Process terminated with status 0 (0 minutes, 0 seconds)
0 errors, 0 warnings (0 minutes, 0 seconds)
\end{lstlisting}

L'exécutable peut maintenant être lancé soit en cliquant sur le bouton Run soit en tapant sur Ctrl-F10.

\hint{L'option F9 (pour Générer et exécuter) combine ces commandes, et peut être encore plus utile dans certaines situations.}

Observez le processus de génération de \codeblocks pour voir ce qui se passe en arrière-plan lors d'une compilation.

\genterm{Suppression d'un fichier}

En utilisant les étapes ci-dessus, ajoutez un nouveau fichier source C++, \textbf{useless.cpp}, au projet. La suppression de ce fichier inutile dans le projet est triviale. Faites tout simplement un clic droit sur \textbf{useless.cpp} dans l'onglet \textbf{Projets} du panneau de \textbf{Gestion} puis sélectionnez \textbf{Enlever le fichier du projet}.

\figures[H][width=.5\columnwidth]{RemoveFile}{Enlever un fichier d'un projet}
 
\hint{Enlever un fichier d'un projet ne le supprime pas physiquement ; \codeblocks l'enlève seulement de la gestion du projet.}


\subsection{Modifier les Options de Génération}
 
Jusqu'ici, les cibles de génération ont été évoquées à plusieurs reprises. Changer entre les 2 versions générées par défaut - \textbf{Debug} et \textbf{Release} - peut simplement se faire via le menu déroulant de la \textbf{Barre d'outils de Compilation}. Chacune de ces cibles qui peut être choisie parmi différents types (par exemple : librairie statique ; application console), contient differentes configurations de fichiers source, variables personnalisées, différents commutateurs de génération (par exemple : symboles de débogage \textit{-p} ; optimisation de la taille \textit{-Os} ; optimisations à l'édition de liens \textit{-flto)}, et bien d'autres options.
 
\figures[H][width=.6\columnwidth]{TargetSelect}{Sélection de Cible}
 
Le menu \menu{Ouvrir un Projet,Propriétés...} permet d'accéder aux propriétés principales du projet actif, \textbf{HelloWorld}. La plupart des configurations du premier onglet, \textbf{Configuration du Projet}, changent rarement. \textbf{Titre} : permet de changer le nom du projet. Si \textbf{Platforme}: est changé en toute autre valeur que celle par défaut \textbf{All}, \codeblocks ne vous permettra de générer que les cibles sélectionnées. Ceci est utile, par exemple, si le code source contient de l'API Windows, et serait donc non valide ailleurs que sous Windows (ou toutes autres situations dépendant spécifiquement du système d'exploitation). \textbf{Makefile}: options qui sont utilisées si le projet doit utiliser un makefile plutôt que le système de génération interne de \codeblocks' (voir \codeblocks et les Makefiles [\pxref{sec:cb_makefiles]} pour plus details).

\genterm{Ajouter une nouvelle cible de génération}

Basculez vers l'onglet \textbf{Générer les cibles}. Cliquez sur \textbf{Ajouter} pour créer une nouvelle cible de génération et nommez-là \textbf{Release Small}. La mise en avant dans la colonne de gauche devrait automatiquement basculer sur la nouvelle cible (si ce n'est pas le cas, cliquez dessus pour changer le focus). Comme la configuration par défaut de \textbf{Type}: - "Application graphique" - est incorrecte pour un programme de type \textbf{HelloWorld}, changez-le en "Application console" via la liste déroulante. Le nom du fichier de sortie \textbf{HelloWorld.exe} est correct sauf que la sortie de l'exécutable se fera dans le répertoire principal. Ajoutez le chemin "bin\osp ReleaseSmall\osp " (Windows) ou "bin/ReleaseSmall/" (Linux) devant le nom pour changer ce répertoire (c'est un chemin en relatif par rapport à la racine du projet). Le \textbf{Répertoire de travail d'exécution}: se rapporte à l'emplacement ou sera exécuté le programme lorsqu'on sélectionnera \textbf{Exécuter} ou \textbf{Générer et exécuter}. La configuration par défaut "." est correcte (elle se réfère au répertoire du projet). Le \textbf{Répertoire de sortie des objets}: doit être changé en "obj\osp ReleaseSmall\osp" (Windows) ou "obj/ReleaseSmall/" (Linux) de façon à rester cohérent avec le reste du projet. \textbf{Générer les fichiers cibles}: pour l'instant, rien n'y est sélectionné. C'est un problème, car rien ne sera compilé si on génère cette cible. Cochez toutes les cases. 

\figures[H][width=0.95\columnwidth]{TargetOptions}{Options de Cible}

L'étape suivante est de changer les paramètres de cible. Cliquez sur \textbf{Options de génération...} pour accéder aux paramètres. Le premier onglet qui s'affiche possède toute une série de commutateurs de compilation (flags) accessibles via des cases à cocher. Sélectionnez "Retirez tous les symboles du binaire" et "Optimiser le code généré pour la taille". Les flags ici contiennent bien d'autres options communément utilisées, cependant, on peut passer outre. Basculez vers le sous-onglet \textbf{Autres options} et ajoutez-y les commutateurs suivants :

\begin{lstlisting}
-fno-rtti
-fno-exceptions
-ffunction-sections
-fdata-sections
-flto
\end{lstlisting}

Maintenant basculer dans l'onglet \textbf{Options de l'éditeur de liens}. La boîte \textbf{Librairies à lier:} vous affiche un bouton pour ajouter diverses librairies (par exemple, \textit{wxmsw28u} pour la version Windows Unicode de la librairie dynamique wxWidgets monolithique version 2.8). Ce programme ne requiert pas de telles librairies. Les  commutateurs personnalisés de l'étape précédente requièrent leur contrepartie lors de l'édition de liens. Ajoutez

\begin{lstlisting}
-flto
-Os
-Wl,--gc-sections
-shared-libgcc
-shared-libstdc++
\end{lstlisting}

dans l'onglet \textbf{Autre options de l'éditeur de liens :}. (Pour plus de détails sur ce que font ces commutateurs, se référer à la documentation de GCC sur les options d'optimisation et les options de l'éditeur de liens.)

\genterm{Cibles Virtuelles}

Cliquez sur \textbf{OK} pour accepter ces changements et retournez au dialogue précédent. Maintenant, vous avez deux cibles de génération "Release", qui auront deux compilations séparées pour lancer \textbf{Générer} ou \textbf{Générer et exécuter}. Heureusement, \codeblocks possède une option pour enchaîner plusieurs générations ensemble. Cliquez sur \textbf{Cibles Virtuelles...}, puis \textbf{Ajouter}. Nommez la cible virtuelle \textbf{Releases} puis cliquez sur \textbf{OK}. Dans la boîte de droite \textbf{Générer les cibles contenues}, sélectionnez les deux \textbf{Release} et \textbf{Release small}. Fermez cette boîte et cliquez sur \textbf{OK} dans la fenètre principale. 

\figures[H][width=.6\columnwidth]{VirtualTargets}{Cibles Virtuelles}

La cible virtuelle "Releases" est maintenant disponible dans la barre d'outils du Compilateur ; la générer produira les sorties suivantes :

\begin{lstlisting}
-------------- Build: Release in HelloWorld ---------------

Compiling: main.cpp
Compiling: hello.cpp
Linking console executable: bin\Release\HelloWorld.exe
Output size is 457.50 KB

-------------- Build: Release Small in HelloWorld ---------------

Compiling: main.cpp
Compiling: hello.cpp
Linking console executable: bin\ReleaseSmall\HelloWorld.exe
Output size is 8.00 KB
Process terminated with status 0 (0 minutes, 1 seconds)
0 errors, 0 warnings (0 minutes, 1 seconds) 
\end{lstlisting}

\end{CREATEPROJECT}

\begin{DEBUGGING}
\section{Débogage avec \codeblocks}\label{sec:debugwithcb}

Cette section décrit comment travailler ne mode débogage.

\subsection{Générer une version "Debug" de votre Projet}

Assurez-vous que le projet soit compilé avec l'option de compilation \textit{-g} (symboles de débogage) activée, et que l'option \textit{-s} (supprimer les symboles) soit désactivée. Ainsi, vous vous assurez que les symboles de débogage sont bien inclus dans l'exécutable.

Les commutateurs d'optimisations du compilateur doivent êtres désactivés, en particulier \textbf{(-s) qui doit} être sur "off".

Gardez à l'esprit que vous devrez peut-être \textbf{re}-générer votre projet car les fichiers objets bien qu'à jour peuvent ne pas avoir été re-compilés avec \textit{-g}. SVP, prenez garde au fait que dans les compilateurs autres que GCC, -g et/ou -s peuvent être des commutateurs différents (-s peut ne pas être du tout disponible).

\menu{Menu,Projet,Options de génération} 
\figures[H][width=0.65\columnwidth]{DbgProjBuildOpt}{Options de Débogage de Génération de Projet}

\subsection{Ajout de Témoins}

\genterm{Dans la version 10.05}
\hint{C'est une très vieille version. Vous ne devriez plus l'utiliser}

Ouvrez la fenêtre des Témoins du Débugueur.

\figures[H][width=0.9\columnwidth]{DbgWatchWindow}{Ouvrir une fenêtre Témoins du débugueur}
La liste des témoins peut être enregistrée dans un fichier et rechargée plus tard. Pour ce faire, clic droit dans la liste des témoins et sélectionnez "enregistrer dans un fichier de témoins" (puis "charger un fichier de témoins" pour les recharger). 
\figures[H][width=0.4\columnwidth]{Save_Watch}{Enregistrer une fenêtre de Témoins}

\genterm{Depuis la version 12.11 et les dernières générations "nightly"}

Dans les dernières générations "nightly" la fenêtre des Témoins a été revue et donc fonctionne différemment de ce qu'il y avait en 10.05.

Actuellement, il y a trois façons d'y ajouter des témoins :

\begin{enumerate}
\item Cliquez sur la dernière ligne (vide) dans la fenêtre des témoins, tapez le nom de la variable (ou une expression complète) puis tapez sur la touche entrée.
\item Quand le débugueur est arrêté sur un point d'arrêt, sélectionnez un nom de variable ou une expression complète, clic droit pour ouvrir un menu de contexte puis sélectionnez "Ajouter un témoin 'expression'".
\item Sélectionnez une expression dans l'éditeur puis glissez-déposez là dans la fenêtre des témoins.
\end{enumerate}

L'inclusion automatique des variables locales et des arguments de fonction ont été ré-implémentés en version 13.12. 

\subsection{Double-clic dans la fenêtre de pile d'Appels}
\hint{Quand on débugue, un double-clic dans une frame de la "pile d'appel" d'une fenêtre de débogage ne met pas à jour automatiquement les variables affichées dans la fenêtre de débogage des "témoins".}

Vous devez effectuer un clic droit dans la frame de la fenêtre de débogage de la "pile d'appel" et sélectionner "Basculer vers cette frame". 
\figures[H][width=1.1\columnwidth]{DWCB_watches_01}{Une Fenêtre de Témoins}

\subsection{Activer des Points d'Arrêt}

Recherchez la ligne contenant la variable que vous voulez observer. Placez un point d'arrêt à un endroit qui vous permettra d'observer la valeur de la variable.

\menu{Menu,Débugueur,Inverser le point d'arrêt}
\figures[H][width=\columnwidth]{DbgSetWatchVar}{Configuration des Variables Témoins}
Lancez le débugueur jusqu'à ce qu'il atteigne le point d'arrêt. Clic droit sur la variable pour configurer un témoin dans la fenêtre des témoins.

Les points d'arrêts peuvent aussi être obtenus ou inversés par un clic gauche dans la marge gauche de l'éditeur. 

\subsection{Notes}
\genterm{Support de Scripts}

\codeblocks utilise en natif le langage de scripts squirrel pour travailler avec gdb, voir: Scripts du Débugueur (\pxref{sec:debugger_scripts}). Depuis la version 7.X, gdb supporte les "pretty printer" de python, et donc, il peut aussi utiliser gdb (avec le support python) pour afficher des valeurs complexes de variables. Voir dans les forum le fil "unofficial MinGW GDB gdb with python released" et "Use GDB python under Codeblocks" pour plus de détails.

\genterm{Débogage d'un fichier seul}

Pour débuguer votre programme vous \textbf{devez absolument} configurer un projet. Les programmes ne consistant qu'en un seul fichier, sans projet associé, ne sont pas supportés.

\genterm{Chemin avec espaces}

Les points d'arrêts ne peuvent pas fonctionner si le chemin et/ou le nom du répertoire où vous avez placé votre projet contient des espaces ou d'autres caractères spéciaux. Pour être sûr du coup, n'utilisez que des lettres anglaises, des chiffres et le caractère '\_'.

\genterm{"Forking"}

Si votre application utilise le système d'appel 'fork' vous aurez des problèmes pour arrêter le programme en cours de débogage ou pour configurer des points d'arrêts à la volée. Voici un lien expliquant les modes forking de GDB : \url{https://sourceware.org/gdb/onlinedocs/gdb/Forks.html}

\genterm{Mise à jour vers une version plus récente de MinGW}

Depuis la version 6.8 de gdb d'Avril 2008, il supporte de nombreuses fonctionnalités qui n'existaient pas dans les versions antérieures. Vous pouvez obtenir une mise à jour en installant les binaires depuis les packages MinGW64 sur SourceForge.
\hint{Le package TDM-Mingw était un bon choix jusqu'à la version 5.1, mais le développement est aujourd'hui abandonné.}

\genterm{Utilisation de CDB 32bit pour des programmes 32bit et CDB 64bit pour des programmes 64bit}

Il semble que le débogage d'un programme 32bit avec CDB 64bit ne fonctionne pas sous Windows 7 (et plus ?), mais CDB 32bit fonctionne parfaitement.

\hint{Ceci ne devrait plus être le cas depuis \codeblocks \codeline{rev>=10920}. Pour plus de détails voir le ticket : \#429}

\genterm{Limites des versions antérieures de MinGW}

Si vous utilisez encore MinGW et gdb 6.7 fourni avec la version 8.02 de \codeblocks, la mise en place de points d'arrêts dans un constructeur ne fonctionnera pas. Voici quelques astuces.

Les points d'arrêt ne fonctionnent pas dans les constructeurs ou les destructeurs dans GDB 6.7 ou toute version antérieure. Cependant, ils fonctionnent dans des routines appelées depuis là. C'est une restriction des versions anciennes de GDB, pas un bug. Alors, vous pouvez faire quelque chose comme : 
\figures[H][width=0.5\columnwidth]{DbgWithCBExp}{Débuguer avec un ancien GDB}
...et placer un point d'arrêt dans "DebugCtorDtor" sur la ligne \codeline{"int i = 0;"}. Le débugueur s'arrêtera sur cette ligne. Si vous avancez alors pas à pas dans le débogage (\menu{Menu Débugueur,Ligne suivante}; ou de façon alternative F7) vous atteindrez le code dans le constructeur/destructeur ("is\_initialized = true/false;"). 

\end{DEBUGGING}

\begin{DEBUGGERSCRIPTS}
\section{Scripts du Débugueur}\label{sec:debugger_scripts}
Cette section décrit les scripts du débugueur.
\subsection{Principe de Base des scripts du débugueur}

\figures[H][width=\columnwidth]{Debuggercommand}{Commande de Débugueur}

Regardez l'image ci-dessus. Cela vous donnera un bref aperçu de comment les scripts du débugueur fonctionnent. Par exemple, vous voulez observer la variable "msg". Il y a deux échanges entre l'extension du débugueur et gdb.

Premièrement, l'extension du débugueur envoie une commande à gdb pour l'interroger sur le type de msg

\begin{lstlisting}
whatis msg
\end{lstlisting}

alors, gdb lui retournera le type

\begin{lstlisting}
type = wxString
\end{lstlisting}

Deuxièmement, le débugger vérifie que wxString est déjà enregistré, puis envoie la commande

\begin{lstlisting}
output /c msg.m_pchData[0]@((wxStringData*)msg.m_pchData-1)->nDataLength
\end{lstlisting}

Puis, gdb répond avec la chaîne ci-dessous :

\begin{lstlisting}
{119 'w', 120 'x', 87 'W', 105 'i', 100 'd', 103 'g', 101 'e', 116 't', 
115 's', 32 ' ', 50 '2', 46 '.', 56 '8', 46 '.', 49 '1', 48 '0', 45 '-', 
87 'W', 105 'i', 110 'n', 100 'd', 111 'o', 119 'w', 115 's', 45 '-', 
85 'U', 110 'n', 105 'i', 99 'c', 111 'o', 100 'd', 101 'e', 32 ' ', 
98 'b', 117 'u', 105 'i', 108 'l', 100 'd'}
\end{lstlisting}

Finalement, la valeur est affichée dans la fenêtre des témoins.

\subsection{Fonctions Script}

Les scripts du débugueur sont semblables à ceux du visualiseur du débugueur de Visual Studio. Ils vous permettent d'écrire un petit bout de code qui sera exécuté par le débugueur dès lors que vous essaierez de regarder un type particulier de variables et peut être utilisé pour afficher du texte personnalisé comportant une information importante dont vous avez besoin.

Remarque dans Game\_Ender - March 23, 2006

\textit{Je ne pense pas qu'il soit possible d'ouvrir une autre fenêtre pour y visualiser quelque chose.}

Regardons maintenant comment cela fonctionne. Tout est à l'intérieur d'un seul fichier placé dans le répertoire des scripts/, dénommé gdb\_types.script :). Le support pour plus de scripts (définis par l'utilisateur) est envisagé dans le futur.


Ce script est appelé par \codeblocks en deux endroits :
\begin{enumerate}
\item quand GDB est lancé. Il appelle la fonction script RegisterTypes() pour enregistrer tous les types définis par l'utilisateur reconnus par le débugueur dans \codeblocks.
\item dès lors que GDB rencontre votre type de variable, il appelle les fonctions script spécifiques de ce type de données (enregistrées dans RegisterTypes() - davantage ci-dessous).
\end{enumerate}

Ceci est un premier aperçu. Regardons en détail le contenu du fichier gdb\_types.script fournit et voyons comment il ajoute le support de \codeline{std::string} dans GDB.

\begin{lstlisting}
// Registers new types with driver
function RegisterTypes(driver)
{
//    signature:
//    driver.RegisterType(type_name, regex, eval_func, parse_func); 

    // STL String
    driver.RegisterType(
        _T("STL String"),
        _T("[^[:alnum:]_]+string[^[:alnum:]_]*"),
        _T("Evaluate_StlString"),
        _T("Parse_StlString")
    );
}
\end{lstlisting}

Le paramètre "driver" est le driver du débugueur mais vous n'avez pas besoin de vous en soucier :) (actuellement, ça ne marche que dans GDB). Cette classe contient une seule méthode : RegisterType. Voici sa déclaration en C++ :

\begin{lstlisting}
void GDB_driver::RegisterType(const wxString& name, const wxString& regex, 
                      const wxString& eval_func, const wxString& parse_func)
\end{lstlisting}

Donc, dans le code du script ci-dessus, le type "STL String" (seulement un nom - qu'importe ce que c'est) est enregistré, fournissant une expression régulière de contrôle sous forme de chaîne de caractères pour l'extension débugueur et, au final, il fournit les noms de deux fonctions indispensables, nécessaires pour chaque type enregistré :

\begin{enumerate}
\item fonction d'évaluation : doit retourner la commande comprise par le débugueur courant (GDB en l'occurrence). Pour "STL string", la fonction d'évaluation retourne une commande de "sortie" de GDB qui sera exécutée par de débugueur GDB.
\item fonction d'analyse : une fois que le débugueur a lancé la commande retournée par la fonction d'évaluation, il passe ses résultats à cette fonction pour des traitements complémentaires. Ce que cette fonction retourne, c'est ce qui est affiché par \codeblocks (habituellement dans la fenêtre des témoins ou dans une fenêtre d'astuces (tooltip)).
\end{enumerate}


Regardons la fonction d'évaluation pour \codeline{std::string}:

\begin{lstlisting}
function Evaluate_StlString(type, a_str, start, count)
{
    local oper = _T(".");

    if (type.Find(_T("*")) > 0)
        oper = _T("->");

    local result = _T("output ") + a_str + oper + _T("c_str()[") 
                   + start + _T("]@");
    if (count != 0)
        result = result + count;
    else
        result = result + a_str + oper + _T("size()");
    return result;
}
\end{lstlisting}

Je ne vais pas expliquer ce que cette fonction retourne. Je vais juste vous dire qu'elle retourne une commande GDB qui fera que GDB imprimera le contenu réel de la \codeline{std::string}. \textit{Oui, vous devrez connaitre votre débugueur et ses commandes avant d'essayer d'étendre ses fonctions.}

Ce que je vais vous dire toutefois, c'est ce que sont les arguments de ces fonctions.

\begin{itemize}
\item type: le type de données, par ex. "char*", "const string", etc.
\item \codeline{a_str}: le nom de la variable que GDB est en train d'évaluer.
\item start: l'offset de départ (utilisé pour les tableaux (ou arrays)).
\item count: le compteur démarrant depuis l'offset de départ (utilisé pour les tableaux).
\end{itemize}

Voyons maintenant la fonction d'analyse utile :

\begin{lstlisting}
function Parse_StlString(a_str, start)
{
    // nothing needs to be done
    return a_str;
}
\end{lstlisting}

\begin{itemize}
\item \codeline{a_str}: la chaîne retournée quand GDB a été exécuté et retournée par la fonction d'évaluation. Dans le cas de \codeline{std::string}, c'est le contenu de la chaîne.
\item start: l'offset de départ (utilisé pour les tableaux).
\end{itemize}

Bon, dans cet exemple, il n'y a rien besoin de faire. "\codeline{a_str}" contient déjà le contenu de \codeline{std:string} et donc il suffit de retourner la chaîne:)

Je vous suggère d'étudier comment wxString est enregistré dans ce même fichier, en tant qu'exemple plus complexe. 

\end{DEBUGGERSCRIPTS}

\begin{MAKEFILES}
\section{\codeblocks et les Makefiles}\label{sec:cb_makefiles}
Cette section décrit comment utiliser un makefile dans \codeblocks en utilisant un exemple wxWidgets.
\subsection{Article du Wiki}

Auteur : Gavrillo 22:34, 21 Mai 2010 (UTC)

Par défaut, \codeblocks n'utilise pas de makefile. Il a ses propres fichiers de projets .cbp qui font la même chose automatiquement. Il y a quelques raisons pour lesquelles vous voudriez utiliser un makefile. Vous êtes peut-être en train de migrer dans \codeblocks un projet qui possède un makefile. Une autre possibilité est que vous voulez que votre projet puisse être généré sans \codeblocks.

Le besoin d'utiliser un pré-processeur n'est pas une raison valable pour utiliser un makefile dans la mesure où \codeblocks a des options de pre/post génération. Depuis le menu \menu{Projet,Options de génération} elles apparaissent dans un onglet avec les étapes de pre/post génération, qui peuvent être utilisées à cet effet.

Ce paragraphe traite plus spécifiquement des makefiles qui utilisent mingw32-make 3.81, CB 8.02 et wxWidgets 2.8 sous Windows Vista, bien qu'il soit presque certain que cela s'applique à d'autres configurations.

Si vous décidez d'utiliser votre propre makefile, vous devez aller sur l'écran de  \menu{Projet,Propriétés} et vous verrez la case à cocher 'ceci est un makefile personnalisé'. Cochez cette case, assurez-vous que le nom placé juste au-dessus est celui que vous voulez pour votre makefile.

Vous devriez aussi regarder dans \menu{Projet,Options de génération}. Il y a un onglet dénommé 'Commandes du Make' (vous avez à déplacer horizontalement les onglets pour tomber dessus). Dans le champ 'Génération du projet/cible' vous devriez voir la ligne \codeline{$make -f $makefile $target}. En supposant que vous êtes en mode débogage, \codeline{$target} sera probablement dénommé 'debug' ce qui n'est pas forcément ce que vous voulez. Vous devriez changer \codeline{$target} par le nom de votre fichier de sortie (avec l'extension .exe et sans le caractère \$ du début).

Un autre ajout utile se trouve dans \menu{Projet,Arbre des projets,Éditer les types et catégories de fichiers}. Si vous y ajoutez makefiles avec le masque \codeline{*.mak} (CB semble préférer .mak plutôt que .mk) vous serez capables d'ajouter votre makefile avec l'extension .mak dans votre projet et il apparaitra dans le panneau de Gestion de projets, sur la gauche.

En supposant que vous voulez éditer le makefile depuis CB, vous devez vous assurer que l'éditeur utilise bien des tabulations (plutôt que des espaces). C'est un problème générique de l'utilitaire make car il a besoin de commencer des lignes de commandes par un caractère tab alors que de nombreux éditeurs remplacent les tabulations par des espaces. Pour obtenir cela dans CB, ouvrez la fenêtre \menu{Paramètres,Éditeur} et cocher la case pour utiliser le caractère de tabulation (tab).

Les problèmes réels commencent toutefois maintenant. La génération automatique de CB ajoute toutes les en-têtes des wxWidgets, mais si vous utilisez un makefile, tout cela n'est pas fait et vous aurez à le faire vous-même.

Heureusement CB possède une autre fonctionnalité qui peut venir à votre secours. Si vous allez dans le menu \menu{Paramètres,Compilateur et Débugueur}, déplacez les onglets horizontalement vers la droite, vous trouverez l'onglet ‘autres paramètres’. Là, cliquez sur la case à cocher  'Enregistrer la génération en HTML ...'. Ceci permettra à CB de créer, au moment de la génération, un fichier HTML qui enregistrera toutes les commandes de génération.

\hint{Cette façon de créer un fichier html de génération n'existe plus dans les versions récentes de CB, mais il y a d'autres solutions}

Si vous compilez (sans utiliser un makefile - donc si vous avez déjà tout remis à plat - désolé) le programme minimal par défaut utilisant wxWidgets, vous pouvez voir les commandes de compilation et d'édition de liens qui produisent ce fichier.

En supposant que vous allez prendre cela comme base pour votre projet, vous pouvez utiliser le contenu du fichier HTML produit comme base de votre makefile.

Vous ne pouvez pas simplement le copier depuis le visualiseur HTML de CB (il n'y a pas cette fonction dans CB) mais vous pouvez charger le fichier dans un navigateur ou un éditeur, et le copier depuis là. Vous le trouverez dans votre répertoire de projet avec \codeline{<le_meme_nom_que_votre_projet\>_build_log.HTML}. Malheureusement, cela requiert encore quelques ajustements comme montrés ci-dessous.

Voici une copie d'un fichier de génération pour un programme wxWidgets de base tel que décrit ci-dessus.

\hint{Pour une meilleure lisibilité, les lignes trop longues ont été découpées. Le signe \codeline{^}  est le séparateur de ligne en mode DOS, le signe \osp \ est le séparateur dans le makefile. Mais vous pouvez avoir les commandes sur une seule ligne à condition d'enlever les séparateurs}


\begin{verbatim}

mingw32-make.exe -f test.mak test.exe

mingw32-g++.exe -pipe -mthreads -D__GNUWIN32__ -D__WXMSW__ -DWXUSINGDLL         ^
    -DwxUSE_UNICODE -Wall -g -D__WXDEBUG__ -IC:\PF\wxWidgets2.8\include         ^
    -IC:\PF\wxWidgets2.8\contrib\include -IC:\PF\wxWidgets2.8\lib\gcc_dll\mswud ^ 
    -c C:\Development\test\testApp.cpp -o obj\Debug\testApp.o

mingw32-g++.exe -pipe -mthreads -D__GNUWIN32__ -D__WXMSW__ -DWXUSINGDLL         ^
    -DwxUSE_UNICODE -Wall -g -D__WXDEBUG__ -IC:\PF\wxWidgets2.8\include         ^
    -IC:\PF\wxWidgets2.8\contrib\include -IC:\PF\wxWidgets2.8\lib\gcc_dll\mswud ^ 
    -c C:\Development\test\testMain.cpp -o obj\Debug\testMain.o

windres -IC:\PF\wxWidgets2.8\include -IC:\PF\wxWidgets2.8\contrib\include       ^
    -IC:\PF\wxWidgets2.8\lib\gcc_dll\mswud -iC:\Development\test\resource.rc    ^ 
    -o obj\Debug\resource.coff

mingw32-g++.exe -LC:\PF\wxWidgets2.8\lib\gcc_dll -o bin\Debug\test.exe          ^
    obj\Debug\testApp.o obj\Debug\testMain.o obj\Debug\resource.coff            ^
    -lwxmsw28ud -mwindows

Process terminated with status 0 (0 minutes, 12 seconds)
0 errors, 0 warnings
\end{verbatim}

Le code ci-dessus peut être converti en un makefile ci-dessous. Il est resté délibérément assez proche de la sortie du fichier HTML.

\begin{verbatim}
# test program makefile

Incpath1 = C:\PF\wxWidgets2.8\include
Incpath2 = C:\PF\wxWidgets2.8\contrib\include
Incpath3 = C:\PF\wxWidgets2.8\lib\gcc_dll\mswud

Libpath = C:\PF\wxWidgets2.8\lib\gcc_dll

flags = -pipe -mthreads -D__GNUWIN32__ -D__WXMSW__ -DWXUSINGDLL     \
        -DwxUSE_UNICODE -Wall -g -D__WXDEBUG__

CXX = mingw32-g++.exe

test.exe : obj\Debug\testApp.o obj\Debug\testMain.o obj\Debug\resource.coff
    $(CXX) -L$(Libpath) -o bin\Debug\test.exe obj\Debug\testApp.o           \
    obj\Debug\testMain.o obj\Debug\resource.coff -lwxmsw28ud -mwindows

obj\Debug\testMain.o : C:\Development\test\testMain.cpp
    $(CXX) $(flags) -I$(Incpath1) -I$(Incpath2) -I$(Incpath3)               \ 
    -c C:\Development\test\testMain.cpp -o obj\Debug\testMain.o

obj\Debug\testApp.o : C:\Development\test\testApp.cpp 
    $(CXX) $(flags) -I$(Incpath1) -I$(Incpath2) -I$(Incpath3)               \
    -c C:\Development\test\testApp.cpp -o obj\Debug\testApp.o

obj\Debug\resource.coff : C:\Development\test\resource.rc
    windres -I$(Incpath1) -I$(Incpath2) -I$(Incpath3)                       \
    -iC:\Development\test\resource.rc -oobj\Debug\resource.coff

# original output from codeblocks compilation
# note I've had to add compiling the .res file
#
# mingw32-g++.exe -pipe -mthreads -D__GNUWIN32__ -D__WXMSW__ -DWXUSINGDLL       ^
#   -DwxUSE_UNICODE -Wall -Wall -g -D__WXDEBUG__                                ^
#   -Wall -g -IC:\PF\wxWidgets2.8\include -IC:\PF\wxWidgets2.8\contrib\include  ^
#   -IC:\PF\wxWidgets2.8\lib\gcc_dll\mswud                                      ^ 
#   -c C:\Development\test\testApp.cpp -o obj\Debug\testApp.o

# mingw32-g++.exe -pipe -mthreads -D__GNUWIN32__ -D__WXMSW__ -DWXUSINGDLL       ^
#   -DwxUSE_UNICODE -Wall -Wall -g -D__WXDEBUG__                                ^
#   -Wall -g -IC:\PF\wxWidgets2.8\include -IC:\PF\wxWidgets2.8\contrib\include  ^
#   -IC:\PF\wxWidgets2.8\lib\gcc_dll\mswud                                      ^
#   -c C:\Development\test\testMain.cpp -o obj\Debug\testMain.o

# mingw32-g++.exe -LC:\PF\wxWidgets2.8\lib\gcc_dll -o bin\Debug\test.exe        ^
#    obj\Debug\testApp.o obj\Debug\testMain.o                                   ^
#    obj\Debug\resource.res -lwxmsw28ud -mwindows

\end{verbatim}

J'ai écrit un makefile générique que je n'ai testé que sous Windows Vista mais qui devrait fonctionner sur tout projet commencé comme décrit ci-dessus. Vous devrez changer le nom du projet et ajuster les chemins appropriés (vous n'aurez probablement qu'à changer Ppath et WXpath).

\begin{verbatim}  

# Generic program makefile
# -- assumes that you name your directory with same name as the project file
# -- eg project test is in <development path>\test\

# Project name and version
Proj := test
Version := Debug

#paths for Project (Ppath) Object files (Opath) and binary path (Bpath)
Ppath := C:\Development\$(Proj)
Opath := obj\$(Version)
Bpath := bin\$(Version)

#Library & header paths
WXpath := C:\PF\wxWidgets2.8
IncWX := $(WXpath)\include
IncCON := $(WXpath)\contrib\include
IncMSW := $(WXpath)\lib\gcc_dll\mswud
Libpath := $(WXpath)\lib\gcc_dll

flags = -pipe -mthreads -D__GNUWIN32__ -D__WXMSW__ -DWXUSINGDLL -DwxUSE_UNICODE \ 
        -Wall -g -D__WXDEBUG__

CXX = mingw32-g++.exe

Obj := $(Opath)\$(Proj)App.o $(Opath)\$(Proj)Main.o $(Opath)\resource.coff

$(Proj).exe : $(Obj)
    $(CXX) -L$(Libpath) -o $(Bpath)\$(Proj).exe $(Obj) -lwxmsw28ud -mwindows

$(Opath)\$(Proj)Main.o : $(Ppath)\$(Proj)Main.cpp
    $(CXX) $(flags) -I$(IncWX) -I$(IncCON) -I$(IncMSW) -c $^ -o $@

$(Opath)\$(Proj)App.o : C:\Development\$(Proj)\$(Proj)App.cpp
    $(CXX) $(flags) -I$(IncWX) -I$(IncCON) -I$(IncMSW) -c $^ -o $@

$(Opath)\resource.coff : C:\Development\$(Proj)\resource.rc
    windres -I$(IncWX) -I$(IncCON) -I$(IncMSW) -i$^ -o $@

.PHONEY : clean

clean:
    del $(Bpath)\$(Proj).exe $(Obj) $(Opath)\resource.coff
\end{verbatim}

\hint{Exporter un makefile depuis un projet \codeblocks est possible indirectement. Vous pouvez l'obtenir à partir de l'utilitaire cbp2make (voir sa description dans \pxref{sec:cbp2make} et/ou un exemple d'utilisation via Tool+ \pxref{sec:tool_cbp2make}.}


Une dernière remarque. Une fois que vous avez utilisé un makefile, toute tentative de compiler sans un makefile vous produira de nombreux warnings. La seule façon que j'ai trouvée pour résoudre cela est de réinstaller CB. Si vous avez besoin des deux façons de compiler il est possible d'installer 2 versions de CB.

\hint{C'est une remarque du texte original dans le Wiki. Mais, en tant qu'utilisateur gd\_on, je n'ai jamais observé ce comportement}

En général, il n'est pas recommandé d'utiliser un makefile, mais s'il y a de bonnes raisons de le faire, alors vous avez ici ce qu'il faut. 

\subsection{Compléments}

Par défaut, \codeblocks génère une cible "Release" et une cible "Debug". Dans votre Makefile, ces cibles peuvent ne pas être présentes. Mais vous avez peut-être une cible "All" (ou "all"). Vous pouvez renommer la cible dans \codeblocks (ou en ajouter une) par ce nom qui a été donné dans le Makefile. 

De plus, votre Makefile génère un exécutable avec un nom spécifique et dans un répertoire spécifique. Dans \codeblocks vous devriez ajuster le chemin et le nom de l'exécutable. Ainsi, \codeblocks, comme il ne connait ni n'analyse le Makefile, trouvera l'exécutable, et la flèche verte d'exécution dans le menu fonctionnera (ou Ctrl-F10).


\end{MAKEFILES}

\begin{CBP2MAKE}
\section{Utilitaire Cbp2make}\label{sec:cbp2make}

Un outil de génération de Makefile pour l'IDE \codeblocks par Mirai Computing. Le texte de cette section provient de son Wiki de cbp2make sur SourceForge.
\hint{Cbp2make n'est pas une extension de \codeblocks, mais une application console autonome, placée dans le répertoire principal de \codeblocks, et qui génère un (ou des) makefile(s) à partir du système de génération interne de \codeblocks}

\subsection{À propos}

"cbp2make" est un outil autonome qui vous permet de générer un (ou des) makefile(s) pour les utiliser via le Make de GNU et en externe aux projets ou aux espaces de travail de l'IDE \codeblocks. (Voir aussi \url{https://forums.codeblocks.org/index.php/topic,13675.0.html]})

\subsection{Utilisation}
\genterm{Création d'un makefile pour un projet unique ou un espace de travail}

Supposons que vous ayez un projet "mon\_projet.cbp" et que vous ayez besoin d'un makefile pour ce projet. Dans le cas le plus simple, tout ce que vous avez à faire c'est :
\begin{lstlisting}
cbp2make -in mon_projet.cbp
\end{lstlisting}

La même chose s'applique pour les espaces de travail.
\begin{lstlisting}
cbp2make -in mes_projets.workspace
\end{lstlisting}

\genterm{Création d'un makefile avec un autre nom de fichier}
Par défaut, "cbp2make" ajoutera l'extension ".mak" au nom du projet pour composer le nom de fichier du makefile.
Si vous voulez change çà, utilisez la commande suivante :

\begin{lstlisting}
cbp2make -lstlisting mon_projet.cbp -out Makefile
\end{lstlisting}

\genterm{Création d'un makefile pour une autre plateforme}
Si vous travaillez sous GNU/Linux et que vous voulez générer un makefile pour Windows ou toute autre combinaison, vous pouvez spécifier la ou les plateformes pour lesquelles vous avez besoin de ces makefiles.

\begin{lstlisting}
cbp2make -in mon_projet.cbp -windows
cbp2make -in mon_projet.cbp -unix
cbp2make -in mon_projet.cbp -unix -windows -mac
cbp2make -in mon_projet.cbp --all-os
\end{lstlisting}
"cbp2make" ajoutera le suffixe ".unix" ou ".windows" ou ".mac" au nom du makefile pour chacune des plateformes respectivement.

\genterm{Création de makefile pour des projets ou espaces de travail multiples}
Si vous avez plus d'un projet ou espace de travail indépendants, vous pouvez les traiter tous à la fois, en ayant recours à un fichier de texte contenant la liste des projets, par ex., "projets.lst", avec un seul nom de projet par ligne.

\begin{lstlisting}
    mon_projet.cbp
    mon_autre_projet.cbp 
\end{lstlisting}

Vous pouvez alors les traiter par la commande :
\begin{lstlisting}
cbp2make -list -in projets.lst
\end{lstlisting}

\subsection{Configuration}

Quelques options spécifiques d'installation ou spécifiques de projet, essentiellement des configurations d'outils, peuvent être enregistrées dans un fichier de configuration. Par défaut (\textit{depuis la rev.110}), cbp2make n'enregistre aucun paramètre dans un fichier de configuration sauf si l'utilisateur spécifie explicitement l'option \codeline{"--config"}. Un fichier de configuration peut être soit global (enregistré dans le profil utilisateur / répertoire home) soit local (enregistré dans le répertoire courant).

SVP, gardez à l'esprit que comme cbp2make est encore à un stade de développement précoce, un ancien fichier de configuration peut devenir incompatible avec la nouvelle version de l'outil et il pourrait être nécessaire de le mettre à jour à la main ou d'en initialiser un nouveau.

\genterm{Initialisation}

\begin{lstlisting}
cbp2make --config options --global
cbp2make --config options --local
\end{lstlisting}

\genterm{Utilisation suivante}

Lorsqu'on invoque cbp2make, il commence par essayer de charger un fichier de configuration local. S'il n'y a pas de fichier de configuration local, il tentera d'en charger un global. Si ces tentatives échouent, la configuration construite en interne est alors utilisée. L'ordre de consultation des configurations peut se changer par les options en ligne de commande \codeline{"--local"} ou \codeline{"--global"}. Si une des options est fournie à cbp2make, la configuration non-spécifiée ne sera pas tentée même si celle spécifiée est absente et que la non-spécifiée existe.

\genterm{Ordre de consultation par défaut}

\begin{lstlisting}
cbp2make -in project.cbp -out Makefile}
\end{lstlisting}

\genterm{Configuration spécifiée explicitement}

\begin{lstlisting}
cbp2make --local -in project.cbp -out Makefile
cbp2make --global -in project.cbp -out Makefile
\end{lstlisting}

\subsection{Syntaxe de la Ligne de Commande}

Génération de makefile :
\begin{verbatim}
cbp2make -in <project_file> [-cfg <configuration>] [-out <makefile>]
[-unix] [-windows] [-mac] [--all-os] [-targets "<target1>[,<target2>[, ...]]"]
[--flat-objects] [--flat-objpath] [--wrap-objects] [--wrap-options]
[--with-deps] [--keep-objdir] [--keep-outdir] [--target-case keep|lower|upper]

cbp2make -list -in <project_file_list> [-cfg <configuration>]
[-unix] [-windows] [-mac] [--all-os] [-targets "<target1>[,<target2>[, ...]]"]
[--flat-objects] [--flat-objpath] [--wrap-objects] [--wrap-options]
[--with-deps] [--keep-objdir] [--keep-outdir] [--target-case keep|lower|upper]
\end{verbatim}

\begin{samepage}
Gestion des outils :
\begin{verbatim}
cbp2make --config toolchain --add \[-unix|-windows|-mac\] -chain <toolchain>
cbp2make --config toolchain --remove \[-unix|-windows|-mac\] -chain <toolchain>
\end{verbatim}
\end{samepage}

Gestion des outils de génération :
\begin{verbatim}
cbp2make --config tool --add \[-unix|-windows|-mac\] -chain <toolchain>
         -tool <tool> -type <type> <tool options>
         
cbp2make --config tool --remove \[-unix|-windows|-mac\] -chain <toolchain>
         -tool <tool>
\end{verbatim}

Types d'outils :      
\begin{verbatim}
    pp=preprocessor as=assembler cc=compiler rc=resource compiler
    sl=static linker dl=dynamic linker el=executable linker
    nl=native linker
\end{verbatim}

Options des outils (communes) :
\begin{verbatim}
    -desc <description> -program <executable> -command <command_template>
    -mkv <make_variable> -srcext <source_extensions> -outext <output_extension>
    -quotepath <yes|no> -fullpath <yes|no> -unixpath <yes|no>
\end{verbatim}

Options des outils (compilateur) :
\begin{verbatim}
    -incsw <include_switch> -defsw <define_switch> -deps <yes|no>
\end{verbatim}

Options des outils (éditeur de liens):
\begin{verbatim}
    -ldsw <library_dir_switch> -llsw <link_library_switch> -lpfx <library_prefix>
    -lext <library_extension> -objext <object_extension> -lflat <yes|no>
\end{verbatim}

Gestion des plateformes :
\begin{verbatim}
cbp2make --config platform \[-unix|-windows|-mac\] \[-pwd <print_dir_command>\]
         \[-cd <change_dir_command>\] \[-rm <remove_file_command>\]
         \[-rmf <remove_file_forced>\] \[-rmd <remove_dir_command>\]
         \[-cp <copy_file_command>\] \[-mv <move_file_command>\]
         \[-md <make_dir_command>\] \[-mdf <make_dir_forced>\]
         \[-make <default_make_tool>\]         
\end{verbatim}

\begin{samepage}
Gestion des variables globales du compilateur :
\begin{verbatim}
cbp2make --config variable --add \[-set <set_name>\] -name <var_name>
        \[-desc <description>\] \[-field <field_name>\] -value <var_value>
        
cbp2make --config variable --remove \[-set <set_name>\] \[-name <var_name>\]
        \[-field <field_name>\]
\end{verbatim}
\end{samepage}

Options de gestion :
\begin{verbatim}
cbp2make --config options --default-options "<options>"    
cbp2make --config show
\end{verbatim}

Options communes :
\begin{verbatim}
cbp2make --local         // utilise la configuration du répertoire courant
cbp2make --global        // utilise la configuration du répertoire home
cbp2make --verbose       // affiche les informations du projet
cbp2make --quiet         // masque tous les messages
cbp2make --help          // affiche ce message
cbp2make --version       // affiche l'information de version
\end{verbatim}

\genterm{Options}

\begin{verbatim}
"Génération de Makefile"

    -in <project_file>   // spécifie un fichier d'entrée ou une liste de fichiers;

    -cfg <configuration> // spécifie un fichier de configuration, voir aussi les
                            options "--local" et "--global";

    -out <makefile>      // spécifie le nom d'un makefile ou une liste de
                            makefiles;

    -unix                // active la génération de makefile compatibles Unix/Linux;

    -windows             // active la génération de makefile compatibles Windows;

    -mac                 // active la génération de makefile compatibles Macintosh;

    --all-os             // active la génération de makefile sur toutes les cibles
                            à la fois;

    -targets "<target1>[,<target2>[, ...]]" // spécifie la seule cible de génération 
                                               pourlaquelle un makefile doit être 
                                               généré;

    --flat-objects       // force les noms "flat" pour les fichiers objets avec un
                            "character set" limité;

    --flat-objpath       // force les noms de chemins "flat" pour les fichiers objets
                            sans sous-répertoires;

    --wrap-objects       // permet l'utilisation de liste d'objets sur plusieurs 
                            lignes ce qui rend un makefile plus facile à lire;

    --wrap-options       // permet l'utilisation de macros sur plusieurs lignes;

    --with-deps          // permet d'utiliser le scanner interne des dépendances 
                            pour les projets C/C++;

    --keep-objdir        // désactive la commande qui supprime les répertoires des
                            fichiers objets dans la cible 'clean';

    --keep-outdir        // désactive la commande qui supprime le répertoire de 
                            sortie des fichiers binaires dans la cible 'clean';

    --target-case [keep|lower|upper] // spécifie un style pour les cibles de 
                                        makefile;
\end{verbatim}

\end{CBP2MAKE}

\begin{INTERNATIONALIZATION}
\section{Internationalisation de l'interface de \codeblocks}\label{sec:cb_Internationalization}

Cette section décrit comment obtenir et utiliser une version internationalisée de \codeblocks.

L'interface de \codeblocks peut être affichée dans votre propre langue. La plupart des chaînes de caractères utilisée en interne pour l'interface de \codeblocks sont introduites par une macro wxWidgets : \_(). Les chaînes qui ne changent pas avec la langue sont normalement introduites par la macro wxT() ou \_T(). Pour obtenir l'interface de \codeblocks affichée dans votre propre langue, vous devez simplement dire à \codeblocks qu'un fichier de langue est disponible. Pour être compréhensible par \codeblocks, ce doit être un fichier .mo obtenu après "compilation" d'un fichier .po. De tels fichiers sont disponibles sur le forum pour le "Français" et sur le site web Launchpad pour un plus large ensemble de langues.

\begin{description}
\item Le site original sur Launchpad est maintenant obsolète : \url{https://launchpad.net/codeblocks }
\item Le sujet du forum traitant de la traduction est \url{https://forums.codeblocks.org/index.php/topic,1022.0.html }. Vous pourrez aussi y trouver des outils d'extraction des chaînes de caractères des fichiers sources de \codeblocks si cela vous intéresse. Ces outils créent un fichier .pot qu'il suffit ensuite de compléter par les traductions afin d'en créer un fichier .po.
\item Un nouveau site web a été créé récemment sur \url{https://launchpad.net/codeblocks-gd }. Il contient plus de 9300 chaînes de caractères alors que l'original n'en avait que 2173! Beaucoup de travail a été fait sur \codeblocks !
\end{description}

Dans la page "translation" choisissez "View All languages", en bas, à droite.

Les anciennes traductions ont été importées dans cette nouvelle page, seulement les langues les plus utilisées (actuellement 14 langues). Sur demande, on peut ajouter des langues (mais les traducteurs auront un peu plus de travail !).\newline
Désolé de cela, mais les noms des traducteurs d'origine ont été perdus dans la plupart des cas  :-[. \newline
C'est la langue Française qui a le plus grand nombre de chaînes traduites. Le modèle (fichier .pot) a été mis à jour sur les versions svn récentes et Launchpad contient le travail de traduction effectué jusqu'à présent. Pour la langue Russe, on a aussi utilisé une page web assez récente mais pas tout à fait à jour. Pas mal de traductions doivent être approuvées, mais je ne suis pas le bon interlocuteur pour ça !\newline
La page launchpad est ouverte en tant que "structured". Donc, vous êtes en mesure de proposer de nouvelles traductions, ou d'en corriger. Dans certains cas, elles devront être approuvées par quelqu'un d'autre avant sa publication.\newline
J'essaierai de maintenir le "modèle" lorsque de nouvelles chaînes en Anglais seront disponibles.

Vous (les traducteurs) devriez être capables de participer. Vous devez seulement posséder (ou créer) un compte launchpad (Ubuntu).

Les autres utilisateurs peuvent demander à télécharger le fichier .po ou .mo. C'est ce dernier (le fichier .mo), la forme binaire, que vous pouvez utiliser pour avoir l'interface de \codeblocks dans votre propre langue : placez le simplement dans votre "répertoire d'installation de codeblocks"/share/CodeBlocks/locale/"language string" (pour moi, sous Windows, c'est\newline
 \file{C:\osp Program Files\osp CodeBlocks\_wx32\_64\osp share\osp CodeBlocks\osp locale\osp fr\_FR}. Ensuite dans le menu Paramètres/Environnement.../Vue vous devriez être capable de choisir la langue.

Quelques détails complémentaires pour utiliser les chaînes traduites dans \codeblocks.

\genterm{Pour les utilisateurs de traductions seulement :}
Téléchargez le fichier au format .mo via la bouton le "requested language". Le nom retourné par launchpad peut être quelque chose comme : de\_LC\_MESSAGES\_All\_codeblocks.mo (ici pour de l'allemand).

Vous devriez mettre ce fichier dans votre répertoire d'installation de codeblocks.

Sous Windows, c'est quelque chose comme :\newline
\file{C:\osp Program Files (x86)\osp CodeBlocks\osp share\osp CodeBlocks\osp locale\osp xxxx} pour une version 32 bits\newline
 ou\newline
\file{C:\osp Program Files\osp CodeBlocks\osp share\osp CodeBlocks\osp locale\osp xxxx} pour une version 64 bits.

Les chemins sous Linux sont assez semblables.

xxxx doit être adapté à votre langue. C'est :
\begin{itemize}
\item de\_DE pour l'Allemand,
\item es\_ES pour l'Espagnol,
\item fr\_FR pour le Français,
\item it\_IT pour l'Italien,
\item lt\_LT pour le Lithuanien,
\item nl\_NL pour le Hollandais,
\item pl\_PL pour le Polonais,
\item pt\_BR pour le Portugais brésilien,
\item pt\_PT pour le Portugais ,
\item ru\_RU pour le Russe,
\item si     pour le Cingalais,
\item zh\_CN pour le chinois simplifié,
\item zh\_TW pour le chinois traditionnel.
\end{itemize}

Créez, si besoin, les sous-répertoires. Puis placez-y votre fichier .mo. Vous pouvez garder le nom du fichier tel que, ou ne garder que les premières lettres (c'est comme vous voulez), mais conservez l'extension .mo.

Puis démarrez \codeblocks. Dans Paramètres/Environnement/Vue vous devez pouvoir cocher la case de la langue (internationalization) puis choisissez votre langue. Si ça ne marche pas, c'est que vous avez probablement oublié quelque chose ou fait une erreur.\newline
Redémarrez \codeblocks pour activer la nouvelle langue.

Si vous voulez retourner à l'anglais, décochez tout simplement la case du choix de la langue.

\genterm{Pour les traducteurs :}
Vous pouvez travailler directement dans launchpad.

\textbf{Problème} : l'interface n'est pas très conviviale.

Vous pouvez aussi télécharger le fichier .po, travailler dessus avec poedit par exemple (la version gratuite suffit). Vous pouvez tester vos traductions en local en la compilant (création d'un fichier .mo) puis en installant ce fichier .mo dans le sous-répertoire adéquat de \codeblocks.

Quand vous aurez suffisamment avancé (c'est votre décision), vous pourrez envoyer ("upload") le fichier .po dans launchpad. Il peut être nécessaire que votre travail soit approuvé ou de le marquer comme à revoir ("to be reviewed").

Ne soyez pas effrayé : c'est un travail assez long. Sur l'ancien site, il y avait 2173 chaînes à traduire. Maintenant il y en a plus de 9300 ! Mais le travail peut être partagé, Launchpad est fait pour ça !

\textbf{Astuce :} Commencez par des menus que vous utilisez souvent : vous verrez les progrès plus vite.


\end{INTERNATIONALIZATION}

\begin{ADDINGLANGUAGES}
\section{Ajout dans le système de génération de \codeblocks d'un support de fichiers non C/C++}\label{sec:cb_AddLanguage}

Cette section descrit comment ajouter dans \codeblocks un support pour d'autres langages que C ou C++. (copie du Wiki: Mandrav Octobre 2007, Mise à jour: MortenMacFly Juin 2012).

\subsection{Introduction}
Comme vous le savez déjà, \codeblocks est adapté principalement au développement en C/C++. Cela signifie que lorsqu'il "voit"  des fichiers C/C++ dans votre projet, il sait comment les compiler et les lier pour en générer un exécutable binaire. Mais qu'en est-il des autres types de fichiers ? Vous pouvez vouloir compiler des fichiers en java ou en python mais, malheureusement, \codeblocks ne sait rien d'eux...\

Et il existe un autre cas : dans le monde réel des projets, il n'est pas rare que certains fichiers appartenant à un projet soient générés automatiquement. Cela se fait via l'utilisation d'une autre programme/script qui éventuellemnt utilise un fichier d'entrée et génère un (ou plusieurs) fichier(s) basé(s) sur cette entrée. \codeblocks, malheureusement, ne sait pas non plus qu'en faire...\

Ou le peut-il ?\

La réponse est : ....... (roulement de tambour) ........ (ta-da) ......... \textbf{Sûr, il le peut !}\

\codeblocks a été mis à jour pour qu'on puisse le configurer pour reconnaitre les fichiers non C/C++ files et y agir en conséquence pendant le processus de génération. Cet article va décrire ces changements et donner un exemple simple mais du monde réel d'une telle utilisation. 

\subsection{Comment ça marche...}

Au cas où vous n'avez jamais regardé les options avancées du compilateur, vous pouvez les trouver en cliquant dans \menu{Paramêtres,Compilateur,Autres paramêtres}. Regarder dans les "Options avancées" en bas à droite, c'est facile de le louper.\

Dans ce dialogue, vous trouverez les lignes de commandes macros utilisées pour générer des fichiers. Par exemple, chaque fichier appartenant au projet, qui a son flag de compilation activé, sera compilé avec la macro dénommée "Compile single file to object file" ("\codeline{$compiler $options $includes -c $file -o $object}", pour les curieux).\


Bien que cela permette de personnaliser la configuration du système de génération, il est clair que cela ne permet pas une personnalisation plus générale.\

Si vous voulez inclure dans votre projet et compiler un fichier java, vous devez définir une commande de génération pour ce fichier particulier, et uniquement pour ce fichier (cliquez avec le bouton droit de la souris sur le fichier dans l'arborescence et choisissez les propriétés). C'est non seulement lourd (imaginez devoir faire cela pour 10 ou 100 fichiers java) mais aussi peu pratique.\

\genterm{...et comment les choses ont évoluées}

La nouvelle fonctionnalité décrite dans cet article vise à supprimer les problèmes décrits ci-dessus et à permettre une plus grande personnalisation du système de génération. Alors, qu'est-ce qui est différent maintenant ? Aller dans \menu{Paramètres, Compilateur, Paramàtres globaux des compilateurs, Autres paramètres} et cliquez sur Options Avancées, vous obtiendrez cette boîte de dialogue : 

\figures[H][width=.55\columnwidth]{AdvancedCompilerOptions}{Options avancées du Compilateur}

Pour commencer, les macros en ligne de commande sont maintenant associées à une liste d'extensions de fichiers sources. Ainsi, chaque macro en ligne de commande (comme le "Compile single file to object file") peut maintenant contenir des macros différentes selon l'extension du fichier source. C'est le cœur de la nouvelle fonctionnalité : en ajoutant une nouvelle paire commande-extension, vous ajoutez effectivement la prise en charge de ces extensions au système de génération !

Une autre chose qui a également été ajoutée est la possibilité de conserver une liste de fichiers que la commande personnalisée va générer (pour chaque paire commande-extension). Ces fichiers générés sont alors automatiquement affichés dans l'arborescence du projet, et font partie du processus de génération, etc. En d'autres termes, ils affectent dynamiquement - et de manière transparente - le projet. Si vous trouvez cela confus, jetez un œil aux exemples ci-dessous et les choses deviendront plus claires :).\

\subsection{Exemples}

\genterm{Voyons déjà un premier exemple}

Voici un exemple concret. J'ai récemment travaillé sur un projet annexe qui m'a demandé d'utiliser SWIG. Ce que fait le programme swig, en termes simples, c'est de prendre en entrée un fichier d'interface spécifique (généralement *.i) et, sur la base de cette entrée, de génèrer un fichier C/C++ à inclure dans votre projet. Cela semble être le scénario idéal à utiliser comme exemple ici :).

Voici ce que j'ai fait : 
\begin{verbatim}
Commande:         Compile single file to object file
Extension:        i
Macro:            swig -c++ -lua $includes -o $file_dir/$file_name.cpp $file
Fichiers générés: $file_dir/$file_name.cpp
\end{verbatim}

Qu'est ce que cela signifie ?

Pour chaque fichier avec l'extension i, utiliser la macro ci-dessus pour le traiter (compiler). Faire aussi savoir à \codeblocks que cette macro va créer un nouveau fichier, dénommé \codeline{$file_dir/$file_name.cpp}.

Avec cette information en main, \codeblocks fera maintenant ce qui suit de manière automatique lorsque vous ajoutez un fichier *.i à un projet :
\begin{itemize}
\item Ajoutera aussi le(s) fichier(s) généré(s) au projet (même s'ils n'existent pas déjà).
\item Affichera le fichier dans une nouvelle arborescence "Auto-generated" (si la catégorisation des fichiers est activée).
\item Comprendra comment traiter (compiler) les fichiers *.i.
\item Programmera également le traitement de tous les fichiers générés (compilation) après le traitement du fichier *.i.
\item Le suivi des dépendances sera maintenu, de sorte que si le fichier *.i est modifié, les fichiers générés seront re-générés également.
\end{itemize}

\genterm{Autre exemple - Ragel}

Compiler une source Ragel State Machine en un fichier C++.
\begin{verbatim}
Commande:         Compile single file to object file
Extension:        rl
Macro:            ragel $file -C -L -o $file.cpp
Fichiers générés: $file.cpp
\end{verbatim}
(Vous devrez vous assurer que l'exécutable ragel est dans votre PATH.)\newline

\genterm{Autre exemple - Bison}

Compilation d'un parseur Bison en fichiers C/C++.
\begin{verbatim}
Commande:         Compile single file to object file
Extension:        y
Macro:            bison -v -d $file -o $file_dir/$file_name.parser.cc
Fichiers générés: $file_dir/$file_name.parser.cc
                  $file_dir/$file_name.parser.hh
\end{verbatim}
(Vous devrez vous assurer que l'exécutable bison est dans votre PATH.)\newline

\genterm{Autre exemple - Flex}

Compilation d'un analyseur de fichiers Flex en fichiers C/C++.
\begin{verbatim}
Commande:         Compile single file to object file
Extension:        l
Macro:            flex -o$file_dir/$file_name.scanner.cc $file
Fichiers générés: $file_dir/$file_name.scanner.cc
\end{verbatim}
(Vous devrez vous assurer que l'exécutable flex est dans votre PATH.)

\genterm{Notes}
    Toutes les commandes par défaut sont associées sans extension. Elles sont utilisées comme solution de repli si une extension correspondante n'est pas définie.

\genterm{Problèmes connus}
\begin{itemize}
\item Actuellement, seules les macros \codeline{$file*} sont supportées comme noms de fichiers générés (\codeline{$file, $file_dir, $file_name et $file_ext}).
\item Si vous changez l'un quelconque des paramètres mentionnés ici dans les options de compilation avancées, vous \textbf{devez} fermer puis ré-ouvrir votre projet pour que les changements soient pris en compte. Pour le moment, aucun message ne le signale.
\item Si vous utilisez un compilateur autre que celui par défaut (pour une compilation croisée, par exemple), vous devrez peut-être effectuer ces réglages dans le compilateur par défaut, et non dans le compilateur croisé, où ils semblent n'avoir aucun effet.
\end{itemize}
\end{ADDINGLANGUAGES}

\begin{VARIABLESTYPES}
\section{Synthèse des types de variables dans \codeblocks}\label{sec:cb_variables_types}

Vous trouverez ici les différents types de variables disponibles dans \codeblocks et quand/comment les utiliser. (recopié du Wiki)

\subsection{Extension Variables d'Environnement}

Ces variables sont propres au système et peuvent être définies ou remplacées par l'extension EnvVars.
C'est utile si vous avez, par exemple, un autre système de génération que \codeblocks qui utilise des variables d'environnement (comme les Makefiles). Ainsi, vous pouvez "partager" cette technologie.
L'extension EnvVars (\pxref{sec:EnvVar_Plugin}) permet de configurer différents jeux de EnvVars que vous pouvez activer ou vous y référer dans les paramètres par projets.
Cela peut être utile, par exemple, pour des paramètres spécifiques à une plate-forme ou des variables de chemin de bibliothèques (sous Linux). 

\subsection{Variables personnalisées globales de Compilateur}

Elles sont utiles, par exemple, pour modifier rapidement un chemin d'accès à une suite de compilateurs.
Par exemple : vous avez installé gcc 10.2.0 et gcc 8.1. Ils ont tous les deux la même structure de chemin d'accès, donc si vous configurez le chemin d'accès principal aux exécutables comme par exemple \codeline{"D:\\Devel\\GCC$(GCC_VER)"} 
et des dossiers include/lib supplémentaires \codeline{"D:\\Devel\\GCC$(GCC_VER)\\include"/"D:\\Devel\\GCC$(GCC_VER)\\lib"}, vous pouvez facilement passer d'un compilateur à l'autre en modifiant simplement la variable personnalisée. \\
Cela s'appliquerait (bien sûr) à *tous* les projets qui utilisent ce même compilateur GCC.

\subsection{Variables personnalisées dans les Options de génération de projet}

Elles sont très utiles si vous voulez compiler votre projet avec deux compilateurs, comme indiqué ci-dessus. 
Vous pouvez avoir deux cibles avec des versions de compilateur différentes qui font toutes deux référence à une configuration de compilateur mais qui ne diffèrent que dans la configuration du chemin. 
En outre, vous pouvez facilement ajouter un "d" aux noms de bibliothèques pour la version de débogage, par exemple wxmsw32ud, un "u" pour une construction unicode, par exemple wxmsw32ud et/ou une chaîne de version pour une version de bibliothèque spécifique, 
par exemple, wxmsw32ud. \newline
Une entrée de bibliothèque dans la configuration du linker qui incorpore les trois exemples ressemblerait à ceci :
\begin{verbatim}
wxmsw$(WX_VER)$(WX_UNICODE)$(WX_DEBUG)
\end{verbatim}

Maintenant vous pouvez configurer les variables du compilateur comme suit :
\begin{verbatim}
WX_VER=32
WX_DEBUG=d
WX_UNICODE=u
\end{verbatim}
pour activer une version unicode, debug v3.2 de la bibliothèque wxWidgets, nommée

\codeline{wxmsw32ud}

Notez que vous pouvez laisser les variables personnalisées vides, donc si vous laissez WX\_DEBUG vide, vous obtiendrez le nom sans débogage

\codeline{wxmsw32u}

(Vous pouvez également omettre la configuration de la variable personnalisée.)

Les valeurs sont remplacées par ordre de détails - les variables personnalisées du compilateur sont remplacées par les variables personnalisées du projet et les variables personnalisées du projet sont remplacées par les variables personnalisées de la cible. Cela n'a de sens que de cette façon... 

\subsection{Où se situent les variables globales dans cet ordre de priorités ?}

Ces variables ont une signification très particulière. Contrairement à toutes les autres, si vous configurez une telle variable et partagez votre fichier de projet avec d'autres personnes qui n'ont *pas* configuré cette variable globale, \codeblocks 
demandera à l'utilisateur de configurer la variable. C'est un moyen très simple de s'assurer que "l'autre développeur" sait ce qu'il doit configurer facilement. \codeblocks demandera tous les chemins nécessaires.\newline
Pour une explication détaillée, veuillez vous référer au paragraphe "Variables globales du compilateur" (\pxref{sec:global_variables}). 

\end{VARIABLESTYPES}

\begin{FILEFORMAT}
\section{Description des Formats de Fichiers}\label{sec:file_formats}

Extraits partiels du Wiki.

Les projets/espaces de travail (workspaces) de \codeblocks sont décrits dans des fichiers XML. Ci-dessous une courte description de chacun d'entre eux.

Cette information a de l'intérêt pour quiconque désirant écrire un importateur/exportateur/générateur pour d'autres systèmes/environnements de génération et par conséquent ajouter un support pour \codeblocks.

\begin{description}
\item[Fichier Espace de Travail] (*.workspace) Définit un espace de travail pour \codeblocks (une collection de projets). Voir les détails ci-dessous (\pxref{sec:workspace_file}) ou dans \url{https://wiki.codeblocks.org/index.php/Workspace_file}.
\item[Fichier Projet] (*.cbp) Définit un projet de \codeblocks. Voir les détails dans \url{https://wiki.codeblocks.org/index.php/Project_file}.
\end{description}

Des fichiers complémentaires on été ajoutés depuis le 12 Décembre 2012 (à partir de la fusion avec la branche XML compiler):

\begin{description}
\item[Fichier de Compilateur] (compiler\_*.xml) Définit une interface vers les compilateurs pour \codeblocks et des procédures d'auto-détection. Voir les détails dans \url{https://wiki.codeblocks.org/index.php/Compiler_file}.
\item[Fichier d'Options de Compilateur] (options\_*.xml) Définit les options et expressions régulières vers les compilateurs pour \codeblocks. Voir les détails dans \url{https://wiki.codeblocks.org/index.php/Compiler_options_file}.
\end{description}

\textit{\codeblocks génère aussi un couple d'autres fichiers (*.layout and *.depend) mais ils ne contiennet que des informations d'état qui ne sont réellement utiles qu'à \codeblocks lui-même.}

\subsection*{Fichier Espace de Travail}\label{sec:workspace_file}

Le fichier Espace de Travail en XML est très simple.

Un espace de travail est une collection de projets. Essentiellement le fichier d'espace de travail fait exactement cela : il décrit un ensemble de projets. Mais voyons le contenu d'un exemple d'espace de travail :

\begin{lstlisting}[language=XML]
<?xml version="1.0" encoding="UTF-8" standalone="yes" ?>
<CodeBlocks_workspace_file>
	<Workspace title="Test Workspace">
		<Project filename="TestConsole/TestConsole.cbp" active="1">
			<Depends filename="TestLib/TestLib.cbp" />
		</Project>
		<Project filename="TestLib/TestLib.cbp" />
	</Workspace>
</CodeBlocks_workspace_file>
\end{lstlisting}

Ce texte en XML définit l'espace de travail dénommé "Test Workspace" contenant deux projets:

\begin{itemize}
\item TestConsole/TestConsole.cbp et
\item TestLib/TestLib.cbp
\end{itemize}

De plus, il définit une dépendance de projet : le projet TestConsole \textit{dépend} du projet TestLib. Cela informe \codeblocks que le projet TestLib doit toujours être généré \textit{avant} le projet TestConsole.

\textbf{NOTE}: \textit{C'est donc une dépendance sur l'ordre de génération qui est configurée. Cela ne forcera \_pas\_ une nouvelle édition de liens de la sortie de TestConsole (qui est un exécutable) si la bibliothèque générée par TestLib est mise à jour. Ce comportement est influencé par un autre paramètre dans le fichier de projet. Voir la description d'un fichier Projet pour ça.}

Bien, j'aurais aimé ajouté quelques commentaires dans le fichier XML lui-même, pour décrire chaque élément, mais comme vous pouvez le voir, c'est assez simple et évident :). La seule chose qui réclame peut-être une explication, c'est l'attribut "active" qu'on voit comme élément de "Project" du projet TestConsole. Cet attribut n'apparaît que lorsqu'il vaut "1" et sur un seul élément "Project" d'un espace de travail. Tout ce qu'il fait c'est de dire lequel des projets de l'espace de travail sera celui actif par défaut, lors de l'ouverture de l'espace de travail dans \codeblocks.


C'est tout. 
\end{FILEFORMAT}
