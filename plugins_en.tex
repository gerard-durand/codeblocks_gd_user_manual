\chapter{Plugins}\label{sec:plugins}

Most of the plugins described ib this chapter are also in the Wiki. Texts and figures have been copied from the Wiki but adapted to be included in Latex documents (Miktex 2.9).

\section{General}

\codeblocks' features can be extend by using plugins. There are generally three types of plugins:
\begin{description}
\item[Core plugins:] developed and maintained by the core C::B team.
\item[Contrib plugins:] developed and maintained by the community and proven to be very valuable. So they are integrated into the C::B SVN.
\item[3rd party plugins:] developed and maintained by the community but not (yet?) in the C::B repository. Theses plugins often have their own repository or are being posted (including the source code) in the forums.
\end{description}

\textbf{If you are looking for plugins}:
\begin{enumerate}
\item Look in the official release. Notice that the installer / package manager might require you to enable some of the plugins specifically. So READ carefully.
\item Search the forums for announcements, especially the forums at \url{http://forums.codeblocks.org/index.php/board,14.0.html}.
\item There might be information on the Wiki concerning other plugins on this page and here : \url{http://Wiki.codeblocks.org/index.php/Announcement_for_plugins/patches}.
\end{enumerate}

For Windows users, the default behavior of the current installer does \textbf{not} install contrib plugins. You need to manually check the "contrib plugin" checkbox when asked for selected components to install. There is no way to install them manually afterwards.


\textbf{If you are developing plugins}: Surely you can work with plugin as you like, but here are some suggestions:

\tab Announce them in the plugin development board in the forums - including the (initial) source code.

OR

\tab Setup your own webpage (or use a file sharing platform) and post the link to the sources/binaries/svn access in the plugin development board in the forums.

OR

\tab Setup a repository, probably at BerliOS or SourceForge, post the link to the sources/binaries/svn access in the plugin development board in the forums. Notice: This is very convenient as attachments in our forum might be deleted from time to time. So it is not safe to post source code in the forums.

THEN

\tab Enter the plugins description on this page.

\tab Announce the plugin using this template on \url{http://Wiki.codeblocks.org/index.php/Template_for_plugin_announcement}

\begin{ASTYLE}
\input{astyle_en}
\end{ASTYLE}

\begin{AUTOVERSIONING}
\input{autoversioning_en}
\end{AUTOVERSIONING}

\begin{BROWSETRACKS}
\input{browsetracks_en}
\end{BROWSETRACKS}

\begin{CODESNIPPETS}
\input{codesnippets_en}
\end{CODESNIPPETS}

\begin{CODECOMPLETION}
\begin{samepage}
\section{Code Completion in \codeblocks}\label{sec:codecompletion}

Two plugins which provide code completion functionality and class browser. They are not compatible with each other. Only one of both can be activated.

\hint {Extracted from Wikipedia: Intelligent code completion is a context-aware code completion feature in some programming environments that speeds up the process of coding applications by reducing typos and other common mistakes. Attempts at this are usually done through auto-completion popups while typing, querying parameters of functions, query hints related to syntax errors. Intelligent code completion and related tools serve as documentation and disambiguation for variable names, functions, and methods.}
\end{samepage}
 
\subsection{Code Completion plugin}

\figures[H][width=.20\columnwidth]{Codecompletion_icon}{Code Completion Icon}

\textbf{CodeCompletion} provides a symbols browser for your projects and code-completion inside the editor.
During code-completion, a system of symbols is used to identify the type associated with the suggested tokens; these symbols are defined in the following table.

\figures[H][width=.75\columnwidth]{CodeCompletion}{Code Completion Table}

Note: This is the user document of Code Completion plugin.
Only C/C++ language is supported by this plugin (currently)...


\subsection{CB Clangd Client}

This plugin provides code completion functionality and class browser by Clangd through LSP (Language Server Protocol).

The home page of this plugin is: \url{https://sourceforge.net/projects/cb-clangd-client/}

The main developer is Pecan.

The related forum discussion is: Code completion using LSP and clangd\newline
(\url{https://forums.codeblocks.org/index.php/topic,24357.msg166136.html})

This documentation is extracted from the wiki: \url{https://wiki.codeblocks.org/index.php/CB_Clangd_Client}

\subsubsection{What is Clangd}

clangd understands your C++ code and adds smart features to your editor:
\begin{itemize}[noitemsep]
\item code completion
\item compile errors
\item go-to-definition
\item go-to-implementation
\item find references
\end{itemize}
and more.

clangd is a language server that can work with your editor via a plugin.\newline
\codeblocks provides Clangd\_client as the needed plugin.

Clangd\_client additionally enhances the clangd server by providing:
\begin{itemize}[noitemsep]
\item call tips
\item function definitions
\item parameter definitions
\item previous and next function positioning
\item symbols browser
\item go-to-file
\item go-to-function
\item symbol renaming
\end{itemize}

\subsubsection{Configuring clangd\_client}\label{sec:cfg_client}

Clangd\_client requires the third-party binary clangd executable.

See \textbf{Windows: Compiler Clangd/LLVM Package Installer} below (\pxref{sec:win_packages}) to install it, or \textbf{Linux: Clangd executable install process} (see \pxref{sec:linux_install})

After a successful clangd executable install, perform the following:

\begin{itemize}[noitemsep]
\item Disable the "CodeCompletion" plugin.
\item Navigate to \menu{Plugins, Manage Plugins} and \textbf{disable} CodeCompletion.
\item Navigate to \menu{Plugins, Manage Plugins} and \textbf{enable} Clangd\_client.
\end{itemize}
\textbf{Restart \codeblocks.}

Tell (or verify) \codeblocks where the clangd executable resides:\par
\begingroup
\leftskip 6ex
Navigate to \menu{Setting, Editor, Clangd\_client, C/C++ parser(tab)} and verify the location of the clangd executable in the box labeled "Specify clangd executable to use".
\par
\endgroup

\subsubsection{Installing Clangd\_client from source or pre-built binary}
\hint {Clangd\_client is now included as a contrib plugin within the "Nightly" builds.
Using a "Nightly" build is the easiest way to update to clangd\_client.
Simply install the "Nightly" and configure as specified below.\\
See the Nightly builds at \url{https://forums.codeblocks.org/index.php/board,20.0.html}
}
\begin{enumerate}[noitemsep]
\item Install the LLVM or Clangd.exe as documented in the section below entitled: \\
          \textbf{Windows Clangd executable install process} (see \pxref{sec:win_install})

\item Disable the Code completion plugin as follows:
    \begin{enumerate}[noitemsep]
    \item Open the Plugin manager via  \codeblocks menu \newline
          \menu{MainMenu, Plugins, Manage plugins...} 
    \item In the Manage Plugin dialog do the following:
        \begin{enumerate}[noitemsep]
        \item Find and select the "Code completion" plugin via it's title 
        \item Press the "Disable" button on the right near the top
        \item If you get any errors please try again.
        \end{enumerate}
    \end{enumerate}
	   
\item Install the Clangd-Client Plugin using one of the following options, which are documented below:
    \begin{enumerate}[noitemsep]
    \item Install via the Plugin Manager
    \item Manually install the plugin files
    \end{enumerate}
	
\item Configure the Clangd-Client Plugin for use as follows:
    \begin{enumerate}[noitemsep]
    \item Select the \codeblocks menu item \menu{Settings, Editor...}
    \item In the list on the left click/select the "clangd\_client" option.
    \item In the "C/C++ parser" tab change the "Specify clangd executable to use" to reference the clangd.exe you installed via step 1) above. \\ 
    Some examples of this could be:
    \begin{verbatim}
    C:\msys64\clang64\bin\clangd.exe
    C:\msys64\clang32\bin\clangd.exe
    C:\LLVM\bin\clangd.exe
    C:\compilers\clang\clangd.exe
    \end{verbatim}
    \end{enumerate}
\end{enumerate}

\subsubsection{Manually Remove Clangd-Client Plugin}

\begin{enumerate}[noitemsep]
\item Exit \codeblocks !
\item If you manually installed the files or used the Plugin manager then you can do the following: 
    \begin{enumerate}[noitemsep]
    \item In the \codeblocks \textbf{\file{...\osp share\osp CodeBlocks}} folder delete the \file{clangd\_client.zip} file
    \item In the \codeblocks \textbf{\file{...\osp share\osp CodeBlocks\osp plugins}} folder delete the \file{clangd\_client.dll} file
    \end{enumerate}
\item If you installed via the Plugin manager then you can delete the files with the following commands:
    \begin{enumerate}[noitemsep]
    \item del \textbf{\file{\%APPDATA\%\osp CodeBlocks\osp share\osp codeblocks\osp plugins\osp clangd\_client.dll}}
    \item del \textbf{\file{\%APPDATA\%\osp CodeBlocks\osp share\osp codeblocks\osp clangd\_client.zip}}
    \end{enumerate}
\item If you want to reuse older code completion remember to re-enable the plugin
\end{enumerate}

\subsubsection{Windows: Clangd executable install process}\label{sec:win_install}

There are three main options to install the clangd.exe:
\begin{enumerate}[noitemsep]
\item Install the LLVM compiler.
\item Manully extract the required files from the LLVM compiler.
\item Install the Clangd package for the Windows compiler you are using if it is available.
\end{enumerate}

The  process for the three options above are detailed below.

\paragraph*{Windows: Install the LLVM compiler}\label{sec:llvm_install}

\begin{enumerate}[noitemsep]
\item Download the latest (non RC/Beta) LLVM Windows executable for your OS (Win32 or Win64) from the following Github LLVM download page: \newline
      \url{https://github.com/llvm/llvm-project/releases} \newline

  As of Jan 2022 the Windows files are names as follows:
  \begin{verbatim}
     LLVM-<version>-win64.exe
     LLVM-<version>-win32.exe
  \end{verbatim}
  where \file{<version>} is the LLVM version, like 13.0.0 or 13.0.1.\\

\item Run the \file{LLVM-<version>-win<xx>.exe} you downloaded to install the LLVM compiler.
\end{enumerate}

\paragraph*{Windows: Manually Extract File from LLVM compiler}\label{sec:llvm_extract}
\begin{enumerate}[noitemsep]
\item Download the latest (non RC/Beta) LLVM Windows executable for your OS (Win32 or Win64) from the following Github LLVM download page:\newline
      \url{https://github.com/llvm/llvm-project/releases} \newline

  As of Jan 2022 the Windows files are names as follows:
  \begin{verbatim}
     LLVM-<version>-win64.exe
     LLVM-<version>-win32.exe
  \end{verbatim}
  where \file{<version>} is the LLVM version, like 13.0.0 or 13.0.1.\\

\item Unzip the \file{LLVM-<version>-win<xx>.exe} file you downloaded using 7ZIP or your prefered ZIP program into a sub-directory
\item Create a new directory to put the clangd.exe and dll's
\item Copy the following files into a the new directory created from the unziped LLVM directory:
    \begin{verbatim}
    bin\clangd.exe
    bin\msvcp140.dll
    bin\vcruntime140.dll
    bin\vcruntime140\_1.dll
    \end{verbatim}
\end{enumerate}

\paragraph*{Windows: Compiler Clangd/LLVM Package Installer}\label{sec:win_packages}         i) \\
   Due to the number of different compilers available for Windows not all of the compilers will have either/both 
   the Clang or LLVM required files.

   If you want to install the specific package(s) for the Windows compiler you are using in order to use it's clangd.exe file please follow the instructions below for the specific compiler you have installed:

   \subparagraph*{MSYS2 Compiler - MinGW64} \hspace{0pt} \\
   There are two main options to install the clangd.exe as follows:
   \begin{enumerate}[noitemsep]
   \item The first option in order to  minimise disk space is to install the Clang extra tools using one of the following packages:       
        {\footnotesize
        \begin{longtable}{|l|l|}\hline
        \textbf{Package}                            & \textbf{Clangd executable}    \\ \hline
%        \endhead   % To repeat the title line, if needed
        mingw-w64-clang-x86\_64-clang-tools-extra   & clang64/bin/clangd.exe        \\
        mingw-w64-x86\_64-clang-tools-extra         & mingw64/bin/clangd.exe        \\ \hline
%        \caption{Msys2 - Clang Extra Packages for MinGW64}
        \end{longtable}
        \par}
 
        To intall the package do the following:
        \begin{enumerate}[noitemsep]
        \item Open the msys2.exe bash shell 
        \item Run the following command: \newline
              \file{pacman -S <Package name in the table above>} \newline
        \end{enumerate}

       "OR" \newline

    \item The second option is to intall the full Clang tool chain as follows:
        \begin{enumerate}[noitemsep]
        \item Open the msys2.exe bash shell 
        \item Run the following command:
              \file{pacman -S mingw-w64-clang-x86\_64-toolchain}
        \end{enumerate}
    \end{enumerate}

    \subparagraph*{MSYS2 Compiler - MinGW32} \hspace{0pt} \\
     There are two main options to install the clangd.exe as follows:
     \begin{enumerate}[noitemsep]
     \item The first option in order to  minimise disk space is to install the Clang extra tools using one of the following packages:
        {\footnotesize
        \begin{longtable}{|l|l|}\hline
        \textbf{Package}                            & \textbf{Clangd executable}    \\ \hline
%        \endhead   % To repeat the title line, if needed
        mingw-w64-clang-i686-clang-tools-extra      & clang32/bin/clangd.exe        \\
        mingw-w64-i686-clang-tools-extra            & mingw32/bin/clangd.exe        \\ \hline
%        \caption{Msys2 - Clang Extra Packages for MinGW32}
        \end{longtable}
        \par}

        To intall the package do the following:
        \begin{enumerate}[noitemsep]
        \item Open the msys2.exe bash shell 
        \item Run the following command:\newline
              \file{pacman -S <Package name in the table above>}\newline
        \end{enumerate}

        "OR" \newline

     \item The second option is to install the full Clang tool chain as follows:
        \begin{enumerate}[noitemsep]
        \item Open the msys2.exe bash shell 
        \item Run the following command:\\
              \file{pacman -S mingw-w64-clang-i686-toolchain}
        \end{enumerate}
    \end{enumerate}

\rule{\textwidth}{0.4pt} \\    
    \textbf{Notes from the \codeblocks forum to avoid mixing incompatible gcc/clangd executables.}\\
\rule{\textwidth}{0.4pt} \\  
    {\small \url{https://forums.codeblocks.org/index.php/topic,24357.msg169412.html#msg169412}}

    \textbf{Don't mix mingw64 with clang64.}
    
    If you are using the gcc from msys2, (the compilers in the folder "\file{msys64\osp mingw64\osp bin}"), you should use "\file{mingw-w64-x86\_64-clang-tools-extra}", (the \file{clangd.exe} is under the folder "\file{msys64\osp mingw64\osp bin}") the same folder as your \file{gcc.exe}.
 
	If you are using the clang tool chain, (the folder "\file{msys64\osp clang64\osp bin}"), you should use "\file{mingw-w64-clang-x86\_64-clang-tools-extra}".

	I found that I just make a big mistake: that is I use the gcc toolchain from "\file{msys64\osp mingw64\osp bin}", but I use the \file{clangd.exe} from "\file{msys64\osp clang64\osp bin}".
    The result is I got a lot of LSP diagnostics messages and errors.\\
    Luckily I found the reason, and fix this issue. Hope this will help other guys.
%}

\subsubsection{Linux: Clangd executable install process}\label{sec:linux_install}

NOTE: Clangd\_client plugin requires a clangd executable version 13 or greater.

Check your current clangd version by running \file{clangd --version}.\newline
If the version is less than 13 you will have to install a newer version.

See \url{https://clangd.llvm.org/installation.html}

Installing the clangd package will usually give you a slightly older version.\newline
Test this by issuing \file{apt-get install --dry-run clangd}

As of 2022/11/16, this suggest that clangd version 10 will be installed.\newline
If the clangd version shows less than version 13, you'll have to install a specific version as follows:

Install a packaged release (\textit{must be version 13 or greater}):

\file{sudo apt-get install clangd-13} (\textit{Must be version 13 or greater}).

This will install clangd as \file{/usr/bin/clangd-13}.

You can now configure clangd\_client plugin by following the above instructions at
\textbf{Configuring clangd\_client} (see \pxref{sec:cfg_client}

If you prefer to install the entire LLVM/Clang package, for convenience, there is
an automatic installation script available that installs LLVM for you.

To install the latest stable version: see \url{https://apt.llvm.org/}, Automatic installation script.
Note that you can specify the version number with this script.
\end{CODECOMPLETION}

\begin{DOXYBLOCKS}
\section{Doxyblocks}\label{sec:doxyblocks}

DoxyBlocks is a plugin for \codeblocks that integrates doxygen into the IDE. It allows you to create documentation, insert comment blocks and run HTML or CHM documents. It also provides configuration of some of the more commonly used settings and access to doxywizard for more detailed configuration.

The settings in the DoxyBlocks toolbar have the following meanings:

\begin{description}
\item[\includegraphics{Doxywizard}] Run doxywizard. Ctrl-Alt-D
\item[\includegraphics{Extract}] Extract documentation for the current project. Ctrl-Alt-E
\item[\includegraphics{Comment_block}] Insert a comment block at the current line. Additionally, DoxyBlocks will try to intelligently read if a method exists on the line for which a comment is being added. Ctrl-Alt-B

\begin{lstlisting}
/** \brief
 *
 * \param bar bool
 * \return void
 *
 */    
void MyClass::Foo(bool bar)
{
    fooBar(bar);
}
\end{lstlisting}

\item[\includegraphics{Comment_line}] Insert a line comment at the current cursor position. Ctrl-Alt-L
\begin{lstlisting}
void MyClass::Foo(bool bar)
{
    fooBar(bar); /**<  */
}
\end{lstlisting}

\item[\includegraphics{Html}] View generated HTML documentation. Ctrl-Alt-H
\item[\includegraphics{Chm}] View generated HTML Help documentation. Ctrl-Alt-C
\item[\includegraphics{Configure}] Open DoxyBlocks' preferences. Ctrl-Alt-P
\end{description}

Doxyblocks works only if doxygen is installed on your system. You need at least the executables doxygen and doxywizard (available in official doxygen distribution at \url{http://www.doxygen.nl/}). Optionally you can have the executable "dot" from the graphviz package (see \url{https://graphviz.gitlab.io/}. On Windows, the help compiler (hhc) may be used to generate chm files.

\genterm{Notes}
\begin{description}
\item In the preferences you have a check box to allow or not allow DoxyBlocks to \textbf{overwrite the doxyfile}. By default, if a doxyfile already exists it is not overwritten to protect any changes that have been made outside DoxyBlocks however this behaviour prevents changes made within DoxyBlocks being written to an existing doxyfile.
\item If a text field is blank in "Preferences", DoxyBlocks will assume that the corresponding executable is available somewhere in your environment's path. You can use macros such as \$(CODEBLOCKS) in your path and they will be expanded automatically.
\item [OUTPUT\_DIRECTORY] Used to specify the (relative or absolute) base path where the generated documentation will be put. If a relative path is entered, it will be relative to the location where doxygen was started. If left blank the current directory will be used. DoxyBlocks will use the path name entered here to create a directory relative to \codeline{<project dir>}. This allows you to create separate doxygen directories for projects that reside in the same directory, or just use a different directory name. If this field is blank, documents will be created in \codeline{<project dir>/doxygen}. Enter directory names without dots, leading separators, volume names, etc. DoxyBlocks does validation on the path name and will strip extraneous characters.
\begin{lstlisting}
Examples:
[blank]           -> <project dir>/doxygen.
"docs"            -> <project dir>/docs.
"docs/sub1/sub2"  -> <project dir>/docs/sub1/sub2.
"doxygen/docs"    -> <project dir>/doxygen/docs.
\end{lstlisting}
\item [OUTPUT\_LANGUAGE] Used to specify the language in which all documentation generated by doxygen is written. Doxygen will use this information to generate all constant output in the proper language. The default language is English. Other languages are supported. 
\item More information in doxygen help files
\end{description}
\end{DOXYBLOCKS}

\begin{EDITORTWEAKS}
\section{Editor Tweaks plugin}\label{sec:editor_tweaks}

The EditorTweaks plugin contains several different features. On a per-file basis, it controls:

\begin{itemize}[noitemsep]
\item word-wrap;
\item line-numbers;
\item tab key emissions (tab characters or spaces);
\item number of spaces the tab key emits;
\item end of line characters (carriage-return + linefeed; carriage-return; linefeed);
\item visibility of end of line characters;
\item on-demand striping of trailing white-space;
\item on-demand synchronization of end of line characters;
\item insert key suppression.
\end{itemize}

From the merge with the Aligner plugin, it has the ability to make sections of code more readable by aligning a specific character.\newline
For example, aligning the "=" in 

\begin{lstlisting}
int var = 1;
int longVarName = 2;
int foobar = 3;
\end{lstlisting}

will result in:

\begin{lstlisting}
int var         = 1;
int longVarName = 2;
int foobar      = 3;
\end{lstlisting}



\end{EDITORTWEAKS}

\begin{ENVVAR}
\section{Environment Variables plugin}\label{sec:EnvVar_Plugin}

From \codeblocks wiki. See also \pxref{sec:EnvVars_Cfg}.

\textbf{Environment variables editor} plugin allows for the setting of system environment variables in the focus of \codeblocks.\newline
A user can have several sets that contain 1..n environment variables.\newline
A user can switch between these sets within the environment variables configuration dialog.\newline
In addition the EnvVars plugin offers an option to projects (within project setup) to apply a certain EnvVar set to activate (and use during compilation).

The dialog for editing the sets is located in \menu{Settings,Environment,Environment variables}.\newline
The dialog for choosing the active set for the current project is located in \menu{Project,Properties,EnvVar options}. 

\textbf{Script binding}

This plugin provides its functionality through a squirrel binding: 

{\footnotesize
\begin{longtable}{|l|l|l|l|}\hline
\textbf{Return value}&\textbf{Name}         &\textbf{Arguments}     &\textbf{Remarks}               \\ \hline
\endhead   % To repeat the title line, if needed
wxArrayString   &EnvvarGetEnvvarSetNames    &                       &Returns all envvars sets       \\
                &                           &                       &available                      \\ \hline
wxString        &EnvvarGetActiveSetName     &                       &Returns the name of the        \\
                &                           &                       & currently active set          \\
                &                           &                       &(from config, /active\_set)    \\ \hline
wxArrayString   &EnvVarGetEnvvarsBySetPath  &const wxString         &Returns the envvars of         \\
                &                           &set\_name              &an envvars set path in         \\
                &                           &                       &the config                     \\ \hline
bool            &EnvvarSetExists            &const wxString         &Verifies if an envvars set     \\
                &                           &set\_name              &really exists in the config    \\ \hline
bool            &EnvvarSetApply             &const wxString\&       &Applies a specific envvar      \\
                &                           &set\_name,             &set from the config            \\
                &                           &bool even\_if\_active  &(without UI interaction)       \\ \hline
void            &EnvvarSetDiscard           &const wxString         &Discards a specific envvar     \\
                &                           &                       &set from the config            \\
                &                           &                       &(without UI interaction)       \\ \hline
bool            &EnvvarApply                &const wxString key,    &Applies a specific envvar      \\
                &                           &const wxString value   &                               \\ \hline
bool            &EnvvarDiscard              &const wxString key     &Discards an envvar             \\ \hline
\caption{Squirrel binding}
\end{longtable}
\par}

\textbf{NOTE}: The value arguments are automatically expanded from macros. You do not have to call ReplaceMacros() on them.

Many other script functions are available. Have a look to \url{https://wiki.codeblocks.org/index.php/Scripting_commands}

\textbf{Example}

On windows in the post or pre build steps:
\begin{lstlisting}
[[EnvvarApply(_("test"),_("testValue"));]]
echo %test%
\end{lstlisting}


\end{ENVVAR}

\begin{FILEMANAGER}
\input{filebrowser_en}
\end{FILEMANAGER}

\begin{HEXEDITOR}
\section{HexEditor}\label{sec:hexeditor}

How a file can be opened in HexEditor within \codeblocks.

\begin{enumerate}
\item \menu{File, Open with HexEditor}
\item Project Navigator context menu (\menu{Open with, Hex editor}
\item Select the Tab Files in the Management Panel. By selecting a file in the FileManager and executing the context menu \menu{Open With Hex editor} this file is opened in HexEditor.
\end{enumerate}

Division of windows:

left is HexEditor view and right is the display as strings

\textbf{Upper line:}
Current position (value in decimal/hex) and percentage (ratio of current cursor position to whole file).

\textbf{Buttons:}

\textbf{Search functions}

\textit{Goto Button:}
Jump to an absolute position. Format in decimal or hex. Relative jump forward or backward by specifying a sign.

\textit{Search:}
Search for hex patterns in the HexEditor view or for strings in the file preview.

\textit{Configuration of the number of columns:}
Exactly, Multiple of, Power of

\textit{Display mode:}
Hex, Binary

\textit{Bytes:}
Select how many bytes should be displayed per column.

\textit{Choice of Endianess:}
BE: Big Endian
LE: Little Endian

\textit{Value Preview:}
Adds an additional view in HexEditor. For a selected value in HexEditor, the value is also displayed as Word, Dword, Float, Double.

\textit{Expression Input:}
Allows you to perform an arithmetic operation on a value in HexEditor. Result of the operation is displayed at the right margin.

\textit{Calc:}
Expression Tester

\textit{Editing a file in the HexEditor:}

Includes Undo and Redo History.

For example, move the cursor to the string view:
Insert spaces with the Insert key.
Delete characters by pressing the Del key.

By entering a text, the existing content is overwritten as a string.

By entering numbers in the HexEditor view the values are overwritten and the preview is updated.


\end{HEXEDITOR}

\begin{INCREMENTALSEARCH}
\section{Incremental Search}

For an efficient search in open files, \codeblocks provides the so-called Incremental Search. This search method is initiated for an open file via the menu \menu{Search,Incremental Search} or by the keyboard shortcut Ctrl-I. The focus is then automatically set to the search mask of the corresponding toolbar. As soon as you begin entering the search term, the background of the search mask will be adjusted in accordance with the occurrence of the term. If a hit is found in the active editor, the respective position in the text is marked in colour. By default the current hit will be highlighted in green. This setting can be changed via \menu{Settings, Editor, Incremental Search} (see \pxref{fig:incremental_search_settings}). Pressing the Return key induces the search to proceed to the next occurrence of the search string within the file. With Shift-Return the previous occurrence can be selected. This functionality is not supported by Scintilla if the incremental search uses regular expressions.

\includegraphics{incremental_search_example}

If the search string cannot be found within the active file, this fact is highlighted by the background of the search mask being displayed in red.

\screenshot{incremental_search_settings}{Settings for Incremental Search}

\begin{description}
\item[ESC] Leave the Incremental Search modus.
\item[ALT-DELETE] Clear the input of the incremental search field.
\end{description}

The icons in the Incremental Search toolbar have the following meanings:

\begin{description}
\item[\includegraphics{incremental_search_clear}] Deleting the text within the search mask of the Incremental Search toolbar.
\item[\includegraphics{incremental_search_previous},\includegraphics{incremental_search_next}] Navigating between the occurrences of a search string.
\item[\includegraphics{incremental_search_highlight}] Clicking this button results in all the occurrences of the search string within the editor being highlighted in colour, instead of only the initial occurrence.
\item[\includegraphics{incremental_search_selected}] Activating this option restricts the search to the text passage marked within the editor.
\item[\includegraphics{incremental_search_matchcase}] This option means a case sensitive search is performed.
\item[\includegraphics{incremental_search_regex}] Regular expression can be used in the input field of incremental search.
\end{description}

\hint{The standard settings of this toolbar can be configured in \menu{Settings,Editor,Incremental Search}.}

%\screenshot{incremental_search_settings}{Settings for Incremental Search}

\end{INCREMENTALSEARCH}

\begin{NASSISHNEIDERMAN}
\section{NassiShneiderman plugin}\label{sec:nassishneiderman}

NassiShneiderman plugin allows the creation of Nassi Shneiderman diagrams within \codeblocks (\cite{url:nassi}). 

\subsection{Create a diagram}

There are two possibilities to create a diagram.

\begin{enumerate}
\item To create an empty diagram select the menu options \menu{File,New,Nassi Shneiderman diagram}.
\item The second option is to creates a diagram from C/C++ source code. 
\end{enumerate}

In an editor window highlight some code to create the diagram from. For example the body of a function/method from the opening braces till the closing brace. Then right click the selection and choose \menu{Nassi Shneiderman,Create diagram} (see \pxref{fig:NassiShneidermanCreate1}). 

\screenshot{NassiShneidermanCreate1}{NassiShneiderman Create}

You should get a new diagram (see \pxref{fig:NassiShneidermanCreate2}).

\screenshot{NassiShneidermanCreate2}{NassiShneiderman Diagram Example}

The parser has some limitations:

\begin{itemize}
\item Comments can not be at the end of a branch.
\item From a definition of a function it is only able to parse the body, not the declaration.
\item For sure, you will find a lot more... 
\end{itemize}

\subsection{Editing structograms}
\subsubsection{What to do with a diagram?}

You can do a lot of things with a structogram:

\begin{enumerate}
\item Store for later usage. The diagram can be stored \menu{File,Save file} or \menu{File,Save file as...}.
\item It is possible to export to different formats \menu{File,Export}
    \begin{itemize}
    \item "Export source..." to save as C source file.
    \item "StrukTeX" to use in your documentation in LaTeX.
    \item "PNG" or "PS" and eventually "SVG" to have a diagram in an image format known by a lot of other tools.
    \end{itemize}        
\item Directly insert the code into the editor: Open or create a diagram. Back in an editor window right click and choose \menu{Nassi Shneiderman,insert from xy} (You get a list with all open diagrams here).
\item Drag'n'Drop the diagram (or parts of it) to other tools. For example to OpenOffice Writer to get an image in your documentation.
\end{enumerate}

If the chosen diagram has a selection, the export or code-generation takes only this part of the diagram. 

\subsubsection{Extensions}

The NassiShneiderman plugin supports some extensions of Nassi-Shneiderman diagrams: 

\begin{itemize}
\item break has a special brick with an "right-arrow"
\item continue has a special brick with a "left-arrow"
\item To be able to create diagrams from c/c++ switch statements, the selection does not implicitly break before a case. The different cases are vertically aligned. Supports C and C++.
\end{itemize}


\end{NASSISHNEIDERMAN}

\begin{LIBFINDER}
\input{lib_finder_en}
\end{LIBFINDER}

\begin{SPELLCHECKER}
\section{SpellChecker plugin}\label{sec:spell_checker}

A plugin to check the spelling of strings and comments.

\subsection{Introduction}
A plugin to check the spelling of strings and comments. The spelling gets checked during typing. Additionally a thesaurus is provided. Both may be accessed on-demand by selecting the word in question, then choosing either Spelling... or Thesaurus... from the Edit menu (the operation can be bound to a hot-key via the Keyboard Shortcuts plugin). The context menu (right click the word) provides spelling suggestions. 

\subsection{Configuration}

Configuration is in the menu \menu{Settings,Editor}. The spell check option are about half way down the list on the left.

\screenshot{ConfigureSpellChecker}{SpellChecker Configuration}

The meaning of the controls are: 
\begin{description}
\item[Enable online spell checker] Enable or disable the spell checker.
\item[Language] The language used for spell checking and the thesaurus is selected by choosing a dictionary. This can also be changed in the status bar.
\item[Path settings, Dictionaries] The plugin is looking in this path for the dictionary files.
\item[Path settings, Thesauri] The plugin is looking in this path for the files needed by the thesaurus.
\item[Path settings, Bitmaps] (Optional) The plugin is looking in this path for the flags to show in the status bar.
\end{description}

\hint{You can use Macros in the above three path settings, such as \$(CODEBLOCKS)/share/codeblocks/SpellChecker. See Variable expansion for more details. This is quite convenient if you use portable \codeblocks.}

\subsection{Dictionaries}

SpellChecker uses a library called hunspell. Hunspell is the spell checker of OpenOffice.org, Mozilla Firefox and other projects. Dictionaries available for those applications are compatible with this plugin.

Open Office provides a collection of dictionaries for several languages and dialects to download. The OOo 3.x extensions (*.oxt) are compressed archives which can be opened with your favourite archiver (for example 7-Zip or File Roller). Copy the .aff file and the .dic file to the directory configured in 'Path settings, Dictionaries' (see above).

If you're running Linux you've probably already got compatible dictionaries installed. Look in /usr/share/hunspell or my choice is /usr/share/myspell/dicts. The reason I like the myspell files is they already include thesaurus files which are named correctly to work with the thesaurus, and everything is all in one location. Don't copy these files. Just point the spell checker to where the files are already located.

I understand on Windows, Firefox and Thunderbird also install compatible dictionary files. These can be found in... \file{C:\osp Program Files\osp Mozilla Firefox\osp dictionaries} or \file{C:\osp Program Files\osp Mozilla Thunderbird\osp dictionaries}. In addition, both OpenOffice.org and LibreOffice install dictionary files to\newline
 \file{C:\osp Program Files\osp (Open/Libre)Office\osp share\osp extensions\osp dict-*}.

The Google Chrome browser also installs dictionaries, but they are .bdic format and the \codeblocks spell checker plugin will not work with them.

\subsection{Thesaurus files}

The files for the thesaurus are also available from OOo, like the dictionaries. Copy the thesaurus files (th\_*.dat and th\_*.idx) to the directory configured in 'Path settings, Thesauri' (see above) and rename them to match the name of the dictionary but prepend "th\_" and let the extension as is.

\textbf{Example}: If the dictionary files (for one language) are "en\_GB.aff" and "en\_GB.dic" the files used for the thesaurus are "th\_en\_GB.idx" and "th\_en\_GB.dat".

On my Linux system I found thesaurus files already installed in /usr/share/myspell/dicts and /usr/share/mythes. Again, don't move the files. Set the spell checker to use the files from their current location.

On Windows, if either OpenOffice.org or LibreOffice is installed, they often include thesaurus files in \file{C:\osp Program Files\osp (Open/Libre)Office\osp share\osp extensions\osp dict-*}. 

\subsection{Bitmaps (flags)}

The bitmap of the actually selected language is shown in the status bar. If no bitmap is found, the name of the language is shown. The bitmap must be a PNG image. Choose a flag from the famfamfam\_flag\_icons and copy it to the directory configured in 'Path settings, Bitmaps' (see above) and rename it to match the name of the dictionary but let the extension png.

\subsection{Styles to check}

Only text with specific styles gets checked (for example only comments and strings). Styles are automatically set by Scintilla (CodeBlocks editing component).

The file OnlineSpellChecking.xml contains a list with indices of the styles to check. The indices differ for different programming-languages so the file contains a list for every programming-language. To add styles, look for the name of the programming-language and the indices in the corresponding lexer\_*.xml file and add this information to the file OnlineSpellChecking.xml.

For example, to check the spelling in bash shell scripts (*.sh files), add the line: 

\codeline{<Language name="Bash" index="2,5,6" />}
\end{SPELLCHECKER}

\begin{SRCEXPORTER}
\input{src_exporter_en}
\end{SRCEXPORTER}

\begin{SVN}
\section{SVN Support}\label{sec:svn}

\hint{This extension is now obsolete. So you'll probably no more find it in recent \codeblocks versions.}

The support of the version control system SVN is included in the \codeblocks plugin TortoiseSVN. Via the menu \menu{TortoiseSVN,Plugin settings} you can configure the accessible svn commands in the tab \menu{Integration}.

\begin{description}
\item[Menu integration] Add an entry TortoiseSVN with different settings in the menu bar.
\item[Project manger] Activate the TortoiseSVN commands in the context menu of the project manager.
\item[Editor] Active the TortoiseSVN commands in the context menu of the editor.
\end{description}

In the plugin settings you can configure which svn commands are accessible via the menu or the context menu. The tab integration provides the entry \menu{Edit main menu} and \menu{Edit popup menu} to configure these commands.

\hint{The File Explorer in \codeblocks uses different icon overlays for indicating the svn status. The TortoiseSVN commands are included here in the context menu.}

\end{SVN}

\begin{TODOLIST}
\input{todo_list_en}
\end{TODOLIST}

\begin{TOOLSPLUS}
\section{Tools+}\label{sec:tools+}

Creating a new tool is fairly simple, and can be completed in a few simple steps. First open \menu{Tools(+),Configure Tools...} to access the User-defined Tools dialog.

\screenshot{tools_setup}{User-defined Tools dialog}

\genterm{Tool Name}

This is the name that will be displayed in the Tools(+) drop-down menu. It will also be displayed as the tab name for tools that redirect to the Tools output window.

\genterm{Command Line}

Any valid command line function and switches can be placed here. Variable substitution is also accepted. The following list contains the more useful variables (see \pxref{sec:builtin_variables} for the full list).

\begin{description}
\item[\$relfile, \$file] Respectively, the relative and absolute name of a selected file.
\item[\$reldir, \$dir] Respectively, the relative and absolute name of a selected directory.
\item[\$relpath, \$path] The relative and absolute name of the selected file or directory.
\item[\$mpaths] A list of selected files or directories (absolute paths only).
\item[\$fname, \$fext] The name without extension and the extension without the name of the selected file.
\item[\$inputstr\{prompt\}] Prompts the user to enter a string of text which is substituted into the command line.
\item[\$if(condition)\{true clause\}\{false clause\}] Resolves to \codeline{false clause} if \codeline{condition} is empty, 0, or false; otherwise \codeline{true clause}.
\end{description}

\genterm{File Types}

Wildcard expressions separated by semicolons will restrict population of the right click menu of a file, directory, or multiple paths in the Project Tree, File Explorer, or Editor Pane to the specified type(s). Leave blank to handle all file/directory types.

\genterm{Working Directory}

The directory from which the command is executed. \codeblocks variables, project variables, and global variables are available. Also,

\begin{enumerate}
\item If \codeline{\$dir} is specified in the command line then \codeline{\$dir} may be used here as well.
\item \codeline{\$parentdir} is available for \codeline{\$relfile}, \codeline{\$file}, \codeline{\$reldir}, \codeline{\$dir}, \codeline{\$relpath}, \codeline{\$path}, \codeline{\$fname}, \codeline{\$fext}, evaluating into the absolute path of the directory containing the item.
\end{enumerate}

\genterm{Tools Menu Path}

Controls the placement of the command in the Tools(+) menu, giving the option of adding sub-menus (multiple levels are allowed).

\begin{itemize}
  \item Submenu/Tool1
  \item Submenu/Tool2
  \item Tool3
\end{itemize}

Will create this structure.

\figures[H][width=.6\columnwidth]{tools_menu_path}{Tools menu structure}

The command name will be used if this entry is blank. If the first character is a period, the command will be hidden.

\genterm{Context Menu Path}

This controls the command's placement in the right-click menu of the Projects and Files tabs of the Management pane. The same rules of structure with the Tools Menu Path apply here.

\figures[H][width=.8\columnwidth]{tools_context_path}{Context menu structure}

Please note that the command will not show up in the context menu unless the Command Line contains one or more of the following: \codeline{\$relfile}, \codeline{\$file}, \codeline{\$reldir}, \codeline{\$dir}, \codeline{\$relpath}, \codeline{\$path}, \codeline{\$fname}, and \codeline{\$fext}.

\genterm{Output to}

This determines where the output of the command will be redirected. The purpose and function of the command will determine which is best to select.
\genterm{Tools Output Window}
Tools that only output results command (and require no input) line generally use this setting. The program will be run invisibly and any output will be redirected to the appropriate tab of the Tools Output Window. The text [DONE] will be added upon the tool's completion.

\figures[H][width=.5\columnwidth]{tool_output}{Tool Output window}

\hint{If the Tools Output window is open when \codeblocks is closed, it may trigger \codeblocks to crash.}

\genterm{\codeblocks Console}
This will cause the program to be run through the executable \file{cb\_console\_runner} (the same program that is launched after Build and run). This is generally used for command line tools with more advanced user interactions, although GUI programs can also be used (especially if the program is unstable and/or also leaves messages in the standard output). Console runner will pause the window (prevent it from closing), display the run time, and the exit code when the program finishes.

\genterm{Standard Shell}
This is the same as placing the command in a batch or shell script, then running it. The program will run in whatever its default method is, and when it finishes, its window will close. This setting is useful for running a program (for example a file or web browser) that must remain open after \codeblocks is closed.

\hint{As the Tools+ plugin is not yet complete, some functionality - specifically Menu Priority and Environment Variables - are not available.}

\subsection{Example Tools}

\genterm{Open file browser to selected file}

\begin{itemize}
\item Windows Explorer
\begin{itemize}
\item Tools Menu
\begin{verbatim}
explorer /select,"$(PROJECTFILE)"
\end{verbatim}
\item Context Menu
\begin{verbatim}
explorer /select,"$path"
\end{verbatim}
\end{itemize}

\item Dolphin
\begin{itemize}
\item Tools Menu
\begin{verbatim}
dolphin --select "$(PROJECTFILE)"
\end{verbatim}
\item Context Menu
\begin{verbatim}
dolphin --select "$path"
\end{verbatim}
\end{itemize}

\hint{The following three Context Menu commands only support folders (but not files).}

\item Nautilus
\begin{itemize}
\item Tools Menu
\begin{verbatim}
nautilus --no-desktop --browser "$(PROJECTDIR)"
\end{verbatim}
\item Context Menu
\begin{verbatim}
nautilus --no-desktop --browser "$dir"
\end{verbatim}
\end{itemize}

\item Thunar
\begin{itemize}
\item Tools Menu
\begin{verbatim}
thunar "$(PROJECTDIR)"
\end{verbatim}
\item Context Menu
\begin{verbatim}
thunar "$dir"
\end{verbatim}
\end{itemize}

\item PCMan File Manager
\begin{itemize}
\item Tools Menu
\begin{verbatim}
pcmanfm "$(PROJECTDIR)"
\end{verbatim}
\item Context Menu
\begin{verbatim}
pcmanfm "$dir"
\end{verbatim}
\end{itemize}
\end{itemize}

\genterm{Update Subversion directory}

\begin{itemize}
\item Windows
\begin{itemize}
\item Tools Menu
\begin{verbatim}
"path_to_svn\bin\svn" update "$inputstr{Directory}"
\end{verbatim}
\item Context Menu
\begin{verbatim}
"path_to_svn\bin\svn" update "$dir"
\end{verbatim}
\end{itemize}

\item Linux
\begin{itemize}
\item Tools Menu
\begin{verbatim}
svn update "$inputstr{Directory}"
\end{verbatim}
\item Context Menu
\begin{verbatim}
svn update "$dir"
\end{verbatim}
\end{itemize}
\end{itemize}

\genterm{Export makefile}

\hint{this uses the command line-tool cbp2make.}\label{sec:tool_cbp2make}

\begin{itemize}
\item Windows
\begin{itemize}
\item Tools Menu
\begin{verbatim}
"path_to_cbp2make\cbp2make" -in "$(PROJECTFILE)"
\end{verbatim}
\end{itemize}

\item Linux
\begin{itemize}
\item Tools Menu
\begin{verbatim}
"path_to_cbp2make/cbp2make" -in "$(PROJECTFILE)"
\end{verbatim}
\end{itemize}
\end{itemize}

\genterm{Compress active project to archive}

\begin{itemize}
\item Windows
\begin{itemize}
\item 7z or zip - Tools Menu (on a single line)
\begin{verbatim}
"path_to_7z\7z" a -t$if(zip == $inputstr{7z or zip?}){zip -mm=Deflate
     -mmt=on -mx9 -mfb=128 -mpass=10}{7z -m0=LZMA -mx9 
     -md=64m -mfb=64 -ms=on} -sccUTF-8 "-w$(PROJECTDIR).."
     "$(PROJECTDIR)..\$(PROJECT_NAME)" "$(PROJECTDIR)*"
\end{verbatim}

\item tar.gz or tar.bz2 - Tools Menu (on a single line)
\begin{verbatim}
cmd /c ""path_to_7z\7z" a -ttar -mx0 -sccUTF-8 "-w$(PROJECTDIR).."
      "$(PROJECTDIR)..\$(PROJECT_NAME)" "$(PROJECTDIR)*" && 
      "path_to_7z\7z" a -t$if(gz == $inputstr{gz or bz2?}){gzip -mx9 
      -mfb=128 -mpass=10 -sccUTF-8 "-w$(PROJECTDIR).." 
      "$(PROJECTDIR)..\$(PROJECT_NAME).tar.gz}{bzip2 -mmt=on -mx9 
      -md=900k -mpass=7 -sccUTF-8 "-w$(PROJECTDIR).." 
      "$(PROJECTDIR)..\$(PROJECT_NAME).tar.bz2}"
      "$(PROJECTDIR)..\$(PROJECT_NAME).tar" && 
       cmd /c del "$(PROJECTDIR)..\$(PROJECT_NAME).tar""
\end{verbatim}

\hint{The Windows command line interpreter has been invoked directly here (\cmdline{cmd /c}), allowing for multiple commands to be chained in a single line. However, this causes the command to fail to execute in the \codeblocks Console.}

\end{itemize}

\item Linux
\begin{itemize}
\item 7z or zip - Tools Menu
\begin{verbatim}
7z a -t$if(zip == $inputstr{7z or zip?}){zip -mm=Deflate -mmt=on -mx9
    -mfb=128 -mpass=10}{7z -m0=LZMA -mx9 -md=64m -mfb=64 -ms=on}
    -sccUTF-8 "-w$(PROJECTDIR).." "$(PROJECTDIR)../$(PROJECT_NAME)"
    "$(PROJECTDIR)*"
\end{verbatim}
\item tar.gz or tar.bz2 - Tools Menu
\begin{verbatim}
tar -cf "$(PROJECTDIR)../$(PROJECT_NAME).tar.$if(gz == $inputstr{gz 
     or bz2?}){gz" -I 'gzip}{bz2" -I 'bzip2} -9' "$(PROJECTDIR)*"
\end{verbatim}
\end{itemize}
\end{itemize}

\end{TOOLSPLUS}

\begin{THREADSEARCH}
\input{thread_search_en}
\end{THREADSEARCH}

\begin{MOREPLUGINS}
\input{more_plugins_en}
\end{MOREPLUGINS}
